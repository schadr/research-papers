\section{Practical Implications for Deployment} 
% [How to integrated our approach in the used repositories and tools to support
% project during development?]

Our approach to mining and constructing social networks, as well as the results
of our studies, have several implications for software practitioners.

\paragraph{Scalable Mining and Analysis Tools}
\ \\
By incrementally mining and using a secondary reporting database (or data
warehouse), we can reduce load and performance problems on the mined software
repositories. Our approach is useful for practitioners who must not impact the
performance of their team's repositories.
% R3.3
This has been proven useful during our extraction of information from the Jazz
repository, which was under constant load due to the global distribution of the
more than 140 Jazz team members.
%
Extracted data, constructed networks, and network analysis measures can be stored
in the data warehouse to avoid extracting or recomputing them multiple times. For
example, the social network using the set of work items and communication around
each build can be stored in the data warehouse and is directly accessible for any
number of further analyses. Computationally intensive analysis and predictions
can be conducted using the data warehouse or by a client accessing the data
warehouse. Once task-based social networks and network analysis measures are
stored in a data warehouse, visualization of social networks and further analysis
becomes efficient and non-intrusive to the users of repositories.

% These networks and measures can assist developers, enabling task-awareness and
% expertise seeking. Additionally, team-leaders and managers can examine social networks and their computed measures
% in the context of project performance data such as integration outcomes.

\paragraph{Team Awareness through Social Network Visualization}
% \ \\ 
Developers working on interdependent tasks can benefit from the visualization of
the social network of the team members who contributed to the interdependent
tasks. The information in the network can include developer contact and
availability, as well as a list of other tasks that they are working on. This
information is particularly important for newcomers to a project, who lack
project specific expertise, such as who has been involved with a particular task,
or in architectural decisions.
% R3.3
We are currently conducting a case study to identify which information is
valuable to support developers and what visualization conveys them most
effectively.
 



% Collaboration scope  focusing on specific teams, inter-team communication, or
% specific task criteria.

\paragraph{Outcome Prediction Using Social Network Analysis}
\ \\
Another application for social network analysis is to predict project outcomes
such as build results based on information about team communication behaviour.
Practitioners can build tools that are embedded in the development environment
that inform developers about the health of upcoming builds. An awareness
notification system that uses a build failure prediction model could indicate
whether the current communication patterns are likely to result in a failed build. 

Building a predictive model from social networks constructed with data mined
from task and task-communication repositories could be done with the following
four steps:

% Note that the steps focus on the integration builds a a certain team and can be
% applied for all teams in parallel.

%\begin{enumerate}

%\item In frequent time intervals (e.g. 30 min.), the social network is
%constructed on the communication about the tasks that are associated with the
%source code changes that are made after the last build. The construction requires
%the querying of the main repositories. The social network measures are computed
%for the network and the resulting measurements and the network are stored and cached in
%the warehouse so that the subsequent network constructions only need to query the
%communication data from the last time interval (e.g. 30 min.).

%\item The trained predictive model that is stored in the warehouse and the
%computed network measurements are used to predict the result of the next upcoming
%build. The prediction result in the warehouse is accessed by the development
%environments to indicate the current prediction result to the developers.

%\item Whenever an integration build finished, an automated process queries the
%remaining new communication that occurred after the last network construction
%from step 2 and the time of the build. The final build related
%communication-based social network and its measurements are constructed, computed
%and stored in the warehouse. The process can either be triggered by notifications
%on subscription or by frequent look-ups.

%\item A new prediction model is trained on all build related networks and
%measurements that are stored in the warehouse. The trained model is stored in the
%warehouse and replaces the previously stored model. The data set that is used to
%train the prediction model increases with each new build and the predictions
%become more precise.
% 
%\end{enumerate}

% To build a predictive one needs to mine the social networks for each build in
% order to populate the data warehouse. All mined social networks of past builds and the associated
% build results are used to train a predictive model. Whereas the social network
% for the upcoming build is only stored in the warehouse for later predictions. 



\begin{enumerate}
\item \textbf{Initialization Predictive Models:}
Initialize the predictive model using the social networks and
outcomes for all existing builds. Store the data on social networks, build
outcomes, and predictive model in a data warehouse for later use.
\item \textbf{Construct Social Network for Upcoming Build:}
Construct the social network for the upcoming build as described
in our approach.
\item \textbf{Predict the Upcoming Build Outcome:}
Calculate network analysis measures for the constructed social
network and use them as input to the predictive model. The model then predicts
an outcome for the upcoming build. At this point, a manager could make
proactive adjustments to attempt to prevent a predicted build failure.
\item \textbf{Update Predictive Model:}
Finally, after the upcoming build has been completed, update the predictive
model to include the social network, network measures, and outcome of the latest
build. The model can then be used again to make an outcome prediction for the
next upcoming build (starting with Step 2).
\end{enumerate}

% Note that steps 2 and 3 are executed repeatedly and only interrupted when a build
% is done in step 4.

%This repository mining and model building approach using task-based
%communication can also be applied to predicting other project-related variables
% such as post-release defects or iteration scheduling accuracy.

If the system indicates a build failure, further analysis could
identify communication deficiencies and prompt managers to rectify the problem
before the build occurs, as described next.
% R3.3
% We are currently conducting a case study to identify how the predictions are
% used by the Jazz team.


\paragraph{Effective Management through Social Network Analysis}
% \ \\

%R0 and R2.6
Beyond visualizations, computed social network measures can be used
by
management as an aid in preventing failed builds. Our research shows that the quality of
communication does matter in the quality of integrations. Monitoring and
affecting communication behavior is a strategy to prevent failures. Software
engineers differ in the expertise and knowledge they bring to collaborative tasks. Team
performance depends not only the information available to developers and the
distribution of knowledge within the team, but also on the communication
structure that facilitates knowledge dissemination within the team. Although we
have no evidence that individual measures of communication structure can
predict integration failure, these measures can be used in identifying
problematic communication behavior likely to result in failed integrations.
Actions that management can take to prevent failures for a
particular team include:

\begin{enumerate}
\item \textbf{Compute Social Network measures or Run Predictive Model early in
the project:} In our study of Jazz, the predictive model performed well even
with the first 25\% of the team communication data (i.e. first quarter of
project timeline). If this is true in other development environments,
practitioners should use this information very early in the project. If the
model predicts error, see next step. Otherwise, computing social network measures such as
density, betweeness and centrality early in the project is an alternative
option that could signal problematic communication behavior in the project.
\item \textbf{Examine team communication structure:}
Identify the presence of patterns of communication that can, when analyzed in
light of project specific technical dependencies, be problematic. An example of
problematic communication structure indicated by social network
measures is that of a team with high coordination needs before an integration
but with a communication network of low density. It is possible that there are too many missing communication links between developers that have
technical dependencies~\cite{ehrlich:2008stc}, or
there is an overall lack of communication in the team. Another example is of a team with several
distributed subteams that need to coordinate and in which a number of
developers have high betweeness but the subteams themselves have little
communication across distance. This may indicate that there are
clusters of developers who should communicate directly, and that
the subteams are only loosely connected and have communication brokers that
may become bottlenecks. 
\item \textbf{Improve communication or knowledge management
procedures:} Adjust communication and knowledge management processes or tools in
the project, to address specific situations identified at previous step. 
If communication links between developers with
coordination needs are missing, assigning communication brokers so that
these developers can better coordinate interdependent work is an
option. Equally valuable, increasing the awareness of these coordination needs
as well as developers' expertise areas in the current project could be done
through regular project meetings or more adequate documentation of these dependencies in project specific knowledge repositories. In other situations, if
communication across distances is relying on information brokers, managers
should consider supporting alternate points of contact in the clusters
identified as relying on these brokers, or enforce documentation of information
that becomes available to everyone in the project. Both these options help
mitigate the risk of information brokers becoming unavailable in the project. 
\end{enumerate}

Despite our intention to be as specific as possible, these 
recommendations for action should consider and be applied in the
particular context of each project. Managers should use the
insights obtained from the examination of communication structures in their
project as a starting point in their further analysis of 
communication in relation to the particular technical and
organizational characteristics of their project.




% Using social network measures, one can
% detect the presence of communication clusters in a distributed project, or
% dysfunctional behavior, such as a lack of communication activity around a
% critical task for an upcoming release.

% On a regular basis, the measures can be reported automatically for the
% project-wide communication network or for specific collaboration scopes. For
% example, this focus can be on tasks related to specific activities (e.g.,
% requirements analysis or maintenance), or certain task properties (e.g., tasks
% within a time range, with high priority, or assigned to one team). Managers and
% team leads monitoring these reports can then maintain awareness of the
% communication structures and possible sources of communication problems within
% these contexts. Detecting problematic communication structures, such as loosely
% connected teams where brokers may become bottlenecks enables project leaders to
% take action and adjust communication processes in a timely and proactive manner.
