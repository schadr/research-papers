% \section{Related Work} The related work consists of three parts. The first is
% focused on how social network were build in the past. The second shows practical
% applications that make use of social networks and the third part gives a quick
% overview over  what has been done with regards to failure prediction.
% 
% \subsection{How To Build Social Networks} Most automatically constructed social
% networks are mined from SCM's~\cite{msr07:weissgerber,msr04:lopez,fse08:pinziger,
% fse08:meneely}. These approaches usually assume that two
% developers that worked together on a module are connected. In this case working
% means that they changed the same module within a pre defined timeframe. This has
% of course the disadvantage that the project specific communication between
% developers is missing. One way to capture part of this communication is to mine
% e-mail archives~\cite{msr06:bird,fse08:bird}. Still this medium is incomplete with
% respect to actual ongoing communication therefore we leverage the discussion
% around work items as provided within \jazztm\. But where were such social networks
% used?
% 
% \subsection{Practical Applications for Social Networks} Social network have been
% used in many different ways. Recent research has manly focused on knowledge
% distribution and failure prediction.
% \begin{description}
% \item[Knowledge.] Efficiently distributing knowledge and fostering awareness in
% teams is a hot topic within software development.
% Several studies have addressed this issue recently. They reach from expertise
% recommending systems within \cite{msr07:minto} and across projects \cite{msr05:ohira}.
% \item[Prediction.] Other studies have leveraged simple networks mined from SCM's
% to predict the existence of post release failures in software modules like
% binaries~\cite{fse08:pinziger,fse08:meneely}.
% \end{description}
% Of course there are more than these application like ... argued the congruence of
% social and technical networks is of use for It governance and de Souza used
% social networks to study the behavior of teams with respect to whom they need and
% want to display their actions.
% 
% \subsection{Failure Predictions} Recent research mostly focused on predicting
% whether a software module (e.g. files or binaries) will fail after shipping a
% product. Most features used to predict this kind of failures leverages the
% technical part like module dependencies, code changes or simply code complexity.
% One of the more recent prediction models by Zimmermann and Nagappan uses social
% network measures on module dependency graphs to predict if a module is
% failure-prone~\cite{icse08:zimmermann}. Others leveraged organizational structures
% evolved around software modules. They connected those modules via developers that
% worked on them to the organizational structure~\cite{icse08:nagappan}. There have been
% studies that go one step further and do not focus on organizational structures
% but look at how developers are connected, mainly those studies like ... and ...
% constructed their networks from SCM systems. These networks connect developers
% through modules they worked on together, which of course misses every other
% communication that went on during that time.
% 
% All in all those approaches focus on failures that will occur after the release
% of a software product since those are most inconvenient for the end user. In
% contrast to that we focus on the product quality during the development. In
% particular on the outcome of builds since if your build does not satisfies the
% minimum quality requirements or does not even build than you cannot ship the
% product. Kim et al. had a similar approach which is more fine grained. They
% predicted if a change introduces a failure into the software \cite{icse07:kim}. This
% method can be used during development but the fine level of detail does not focus
% on the important part namely predicting if you reached a deployable state or not.
