\section{Conclusions}
Task-based social network mining and analysis enables you, as a participant in a
software project, to tap into a vast pool of otherwise inaccessible communication
information. Our approach has been applied to two scenarios with the Jazz project
illustrating its applicability and gleaning new insights into the social patterns
of that team. The use of social networks in software engineering is relatively
unexplored and holds much promise for future applications.

%  Mining repositories and social networks gained focus by researchers to analyze
% software development projects and the involved technical and human
% relationships. While researchers published impressive findings, few of the
% techniques are integrated and used in the development environment to directly
% support the project. We provide an approach of mining task-related
% communication-based social networks for any repositories that support \people
% s, \cu s, and communication. The feasibility of the approach is shown by two
% studies mining the Jazz repository of the Jazz development project. We propose
% to utilize social network analysis to directly support the mined project. A
% variety of practical applications in the area of reporting or awareness
% applications can support the daily work of project participants by integrating
% the mining process into the development environment. We propose initial steps
% to realize the integration without affecting the performance of the main
% repositories used.
