Suppose you are the manager of a software team and are responsible for delivering
a software product on a specific date. Your team uses a build system to integrate
their work before delivery. When the build fails, your team needs to spend extra
time diagnosing the integration issue and reworking code. As the manager, you
suspect that your team failed to communicate about a code dependency which broke
the build. Your team needs to communicate in timely manner to propagate
information about their interdependent work to achieve a successful integration
build. How can you understand the communication of your team? Social network
analysis can give you insight into the communication patterns of your team that
may have been the cause of the build failure.

%R3.4
% Visualization of social networks can provide insight into the communication
% patterns of teams working on a feature, or at a particular geographic location.
% Analysis of social network structures can also assist management in altering
% organizational configurations to meet specific collaboration goals. For example,
% the fact that two project members do not effectively communicate about their
% interdependent work may be due to geographic distance. Managers may decide to
% establish new communication channels or assign communication brokers so that
% distributed teams can better coordinate interdependent work.

Current and timely knowledge of the social network of people in your project,
whether you are a project manager, a team leader, or a developer is important in
many situations beyond broken builds. Knowledge of project experts and central
communicators is 
 information that is often invisible in the development
environment. The distributed nature of software development, compounded with the
typical high turnover in outsourcing relationships, only adds to this problem.
Highly interdependent teams often need to function across organizational and
geographic boundaries and face significant challenges to maintain awareness and
effectively communicate with their team. Examination of social networks can
identify collaboration problems such as missing communication links between
interdependent team members. Since newcomers specifically lack historical project
information, they may benefit from insights into the project's social networks
that expose expertise and active communicators.

%R3.4
However, constructing an explicit representation of social networks within an
organization is not trivial. Communications play a main role in social networks
within organizations. The difficulty stems from the fact that people in software
projects communicate through diverse channels some of which are not easily
recordable. Given the difficulty of capturing and recording face-to-face and
telephone conversations, software project repositories, such as bug databases,
source code repositories, and automated build systems provide rich sources for
mining developers' communication. The recorded communication artifacts must be
translated into meaningful conversations about tasks of interest. How do you
leverage developers' communication that is \emph{related} to a specific
collaborative task? How can you construct a social network of people that
collaborate on a task that is of interest to you?

% DD. addressed comment #
Recent research in software engineering has used social networks to study
collaboration in software teams and mined data from different repositories, such
as software and email
archives~\cite{msr06:bird,herbsleb:2008fse,ehrlich:2008stc}. The social networks
developed in these works are, however, difficult to compare, having been
constructed for different research purposes. The research contribution of this
paper is a systematic approach to mine large software repositories to generate
social networks that use task-based communication between developers. This
approach is independent of any specific repository and can be applied in any
project that stores collaborative tasks and related communication information. We
also describe our experiences in using this approach in our research and discuss
practical implications for deployment. When applied to mine the software
repository of the IBM Rational Jazz project, the described approach allowed us to
discover that properties of developer social networks can be used to predict
integration build results. As well, we found that the large Jazz team experienced
less delay in communication than expected due to the distributed nature of the
project. In addition, they exhibited a highly connected project-wide social
network with effective information distribution among seven geographic sites.



% The project also exhibited a highly connected project-wide social
% network with effective  between seven development sites.

%XXX maybe here, if needed, a brief summary of what the results of
%our investigation of SN in \jazztm\ were, to whet the reader's appetite.