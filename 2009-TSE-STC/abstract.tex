Socio-technical congruence is an approach that measures coordination by examining the alignment between the technical dependencies and the social coordination in the project.
We conduct a case study of coordination in the IBM\textsuperscript{\textregistered} Rational Team Concert\textsuperscript{\textregistered} project, which consists of 151 developers over seven geographically distributed sites, and expect that high congruence leads to a high probability of successful builds.
We examine this relationship by applying two congruence measurements: an unweighted congruence measure from previous literature, and a weighted measure that overcomes limitations of the existing measure. We discover that there is a relationship between socio-technical congruence and build success probability, but only for certain build types, and observe that in some situations, higher congruence actually leads to lower build success rates.
We also observe that a large proportion of zero-congruence builds are successful, and that socio-technical gaps in successful builds are larger than gaps in failed builds.
Analysis of the social and technical aspects in IBM\textsuperscript{\textregistered} Rational Team Concert\textsuperscript{\textregistered} allows us to discuss the effects of congruence on build success.
Our findings provide implications with respect to the limits of applicability of socio-technical congruence and suggest further improvements of socio-technical congruence to study coordination. 
