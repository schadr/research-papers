\section{Related Work}
\label{sec:relwork}
Our study aims on integrating work investigating team collaboration and failure prediction to produce actionable knowledge upon which developer can act.
Several studies bear relevance with respect to different dimensions of our work:

\emph{With respect to research on software builds:}
To the best of our knowledge the studies by Hassan et al.~\cite{hassan:ase:2006}
and Wolf et al.~\cite{wolf:icse:2009} are the only studies that conducted
research to predict build outcome. Hassan et al.~\cite{hassan:ase:2006} found
that a combination of social metrics (e.g. number of authors) and technical
metrics (e.g. number of code changes) derived from the source code repository
yield to be best predictor. On the other hand Wolf et al.~\cite{wolf:icse:2009}
solely used metrics that they derived from the social network created from
discussions among developers and showed that communication structure has an
influence on the build outcome.

\emph{With respect to team coordination:}
%Coordination is defined as ``integrating or linking together different parts of
%an organization to accomplish a collective set of tasks''~\cite{vandeven1976}. 
In
order to manage changes and maintain quality, developers must coordinate. In
software development, coordination is largely achieved through communicating with
people who depend on the work that you do \cite{kraut1995:coordination}. The
software engineering literature is recognizing the role of communication as
something that should be nurtured not eliminated and recent
collaborative software development environments aim to support developers'
social interactions along with artifact creation activities~\cite{nakakoji2010:rdc}.

%While a failed build is not necessarily a disaster, it slows down work significantly and is considered extremely undesirable.
%A build result thus serves as an indicator of the health of the software project up until that point in time. If the developers successfully coordinate the integration of code between the previous build and the upcoming build, then the build should succeed.

Ehrlich et al.~\cite{ehrlich:icgse:2006} investgiated how social networks can be
used to leverage knowledge in distributed teams. Backstrom et
al.~\cite{backstrom:kdd:2006} took a more general approach and investigated the
evolution of large social networks and the information they hold. Chung et
al.~\cite{chung:cpr:07} reported in recent work about behavior of individuals
while performing knowledge intensive tasks. There have been a number of studies
that investigated communication structures to identify good
coordination practices
(e.g.~\cite{hinds:cscw:2006,hossain:cscw:2006,bird:fse:2008,hinds:hicss:2008}). In contrast to studies of the general development process, Marczak studied social
networks to identify best practices for requirements management
processes~\cite{marczak:re:2008}.

Inspired by Conways Law~\cite{conway:datamination:1968}, Cataldo et
al.~\cite{cataldo:cscw:2006,cataldo:esem:2008} formulated a coefficient that
measures the alignment of the social and technical networks defining the term of
socio-technical congruence. They observed that higher socio-technical congruence
leads to higher developer
productivity~\cite{cataldo:cscw:2006,cataldo:esem:2008}. Others used this
notion and coefficient to further investigate the effect of congruence
(e.g.~\cite{valetto:msr:2007}). Prior to Cataldo et
al.~\cite{cataldo:cscw:2006,cataldo:esem:2008} proposal,
Ducheneaut~\cite{ducheneaut:cscw:2005} investigated the evolution of social and
technical relationships of open source project participants to see how those
participants become a part of the community.


\emph{With respect to failure prediction:}
There have been a large number of studies looking into predicting failures. For
example, Zimmermann et al.~\cite{zimmermann:icse:2008} used
networks constructed from interdependent binaries to predict the failure
probability of files. They used different metrics characterizing the relationship
binaries have to each other and found that ego centric social network measures
are powerful failure predictors. Previous research conducted at Microsoft used
code complexity metrics, such as cyclomatic complexity or object oriented
metrics, that are derived from source code. Nagappan et
al.~\cite{nagappan:icse:2006} found
that no single source code metric was capable of being a good
predictor over all studied Microsoft projects. Motivated by this study that
suggested that predictions might be domain specific Schr\"oter et
al.~\cite{schroeter:isese:2006} characterized domain of packages in the Eclipse
project by their imports and found them to be a powerful predictor.


\emph{With respect to failures related to team coordination:}
More recent studies started to relate the social with the technical
dimensions of software development to build predictive models. Pinzger et
al.~\cite{pinzger:fse:2008} successfully used social networks connecting
developers via code artifacts to predict failures. Meneely et
al.~\cite{meneely:fse:2008} used similar networks but excluded the code artifacts
and connected the developers directly. Two studies at Microsoft looked into the
geographical~\cite{bird:acm:2009} and organizational~\cite{nagappan:icse:2008}
distance between people that worked on the same binary and the relation to the
failure proneness of said binary. They found that the organizational distance is
a very powerful predictor of failure proneness of binaries whereas the
investigation of geographical distance has little to no effect. A recent
study~\cite{bird:issre:2009} combines the work of Pinzger et
al.~\cite{pinzger:fse:2008} and
Zimmermann~\cite{zimmermann:icse:2008} by creating
socio-technical networks that capture developer contributions and binary
interdependencies. They found this combination to be a more powerful predictor
that works for different software project and even prevails across multiple
revisions of a project.

% talk about the short coming of the failure prediction models
Despite the high predictive power of the state-of-the-art prediction models, most
of them suffer from a profound shortcoming: the knowledge provided by these
models is not always easily actionable. In this work we lie the foundation to
the recommender system described by Schr\"oter et
al.~\cite{schroeter:rsse:2008}. 
%
This recommender system compares social networks derived from technical and social dependencies over successful and failed builds to recommend changing failure related pairs of developers.
%
%AND WHICH\ldots A BIT MORE HERE ABOUT HOW THIS
%RECOMMENDER SYSTEM MAY WORK BASED ON THE RESULTS OF THIS WORK (YOU CAN REPLACE
%THE NEXT SENTENCE WITH THIS DETAIL IF SPACE IS A PROBLEM). 
%
This way we generate knowledge that not only tells us when a build fails but immediately helps us to provide suggestions to prevent the build from failing.

