% This is "sig-alternate.tex" V1.9 April 2009
% This file should be compiled with V2.4 of "sig-alternate.cls" April 2009
%
% This example file demonstrates the use of the 'sig-alternate.cls'
% V2.4 LaTeX2e document class file. It is for those submitting
% articles to ACM Conference Proceedings WHO DO NOT WISH TO
% STRICTLY ADHERE TO THE SIGS (PUBS-BOARD-ENDORSED) STYLE.
% The 'sig-alternate.cls' file will produce a similar-looking,
% albeit, 'tighter' paper resulting in, invariably, fewer pages.
%
% ----------------------------------------------------------------------------------------------------------------
% This .tex file (and associated .cls V2.4) produces:
%       1) The Permission Statement
%       2) The Conference (location) Info information
%       3) The Copyright Line with ACM data
%       4) NO page numbers
%
% as against the acm_proc_article-sp.cls file which
% DOES NOT produce 1) thru' 3) above.
%
% Using 'sig-alternate.cls' you have control, however, from within
% the source .tex file, over both the CopyrightYear
% (defaulted to 200X) and the ACM Copyright Data
% (defaulted to X-XXXXX-XX-X/XX/XX).
% e.g.
% \CopyrightYear{2007} will cause 2007 to appear in the copyright line.
% \crdata{0-12345-67-8/90/12} will cause 0-12345-67-8/90/12 to appear in the copyright line.
%
% ---------------------------------------------------------------------------------------------------------------
% This .tex source is an example which *does* use
% the .bib file (from which the .bbl file % is produced).
% REMEMBER HOWEVER: After having produced the .bbl file,
% and prior to final submission, you *NEED* to 'insert'
% your .bbl file into your source .tex file so as to provide
% ONE 'self-contained' source file.
%
% ================= IF YOU HAVE QUESTIONS =======================
% Questions regarding the SIGS styles, SIGS policies and
% procedures, Conferences etc. should be sent to
% Adrienne Griscti (griscti@acm.org)
%
% Technical questions _only_ to
% Gerald Murray (murray@hq.acm.org)
% ===============================================================
%
% For tracking purposes - this is V1.9 - April 2009

\documentclass{sig-alternate}
%\documentclass{acm_proc_article-sp}

\usepackage{comment}
\usepackage{amsmath}
\usepackage{amsfonts}
\usepackage{amssymb}
\usepackage{array}
\usepackage{booktabs}
\usepackage{capt-of}
\usepackage{colortbl}
\usepackage{graphicx}
\usepackage{multirow}
\usepackage{subfig}
%\DeclareCaptionType{copyrightbox}
\usepackage{pifont}
\usepackage[latin1]{inputenc}
\usepackage{times}
\usepackage{url}
\usepackage{boxedminipage}
\usepackage{xspace}
\usepackage{sepnum}
\usepackage{cite}
\usepackage{fancyhdr}

\pretolerance=10000
%\overfullrule3pt

\newenvironment{note}%
{\medskip
\noindent
\vspace{2pt}
%\setlength{\fboxsep}{4pt}
%\setlength{\leftmargini}{4pt}
\let\emph=\textbf
\begin{boxedminipage}{.99\columnwidth}\em}%
{\end{boxedminipage}}

\begin{document}
%
% --- Author Metadata here ---
\conferenceinfo{WOODSTOCK}{'97 El Paso, Texas USA}
%\CopyrightYear{2007} % Allows default copyright year (20XX) to be over-ridden - IF NEED BE.
%\crdata{0-12345-67-8/90/01}  % Allows default copyright data (0-89791-88-6/97/05) to be over-ridden - IF NEED BE.
% --- End of Author Metadata ---

\title{To Talk or Not to Talk: Factors that Influence Communication around Changesets}
%
% You need the command \numberofauthors to handle the 'placement
% and alignment' of the authors beneath the title.
%
% For aesthetic reasons, we recommend 'three authors at a time'
% i.e. three 'name/affiliation blocks' be placed beneath the title.
%
% NOTE: You are NOT restricted in how many 'rows' of
% "name/affiliations" may appear. We just ask that you restrict
% the number of 'columns' to three.
%
% Because of the available 'opening page real-estate'
% we ask you to refrain from putting more than six authors
% (two rows with three columns) beneath the article title.
% More than six makes the first-page appear very cluttered indeed.
%
% Use the \alignauthor commands to handle the names
% and affiliations for an 'aesthetic maximum' of six authors.
% Add names, affiliations, addresses for
% the seventh etc. author(s) as the argument for the
% \additionalauthors command.
% These 'additional authors' will be output/set for you
% without further effort on your part as the last section in
% the body of your article BEFORE References or any Appendices.
\numberofauthors{3} %  in this sample file, there are a *total*
% of EIGHT authors. SIX appear on the 'first-page' (for formatting
% reasons) and the remaining two appear in the \additionalauthors section.
%
\author{
% You can go ahead and credit any number of authors here,
% e.g. one 'row of three' or two rows (consisting of one row of three
% and a second row of one, two or three).
%
% The command \alignauthor (no curly braces needed) should
% precede each author name, affiliation/snail-mail address and
% e-mail address. Additionally, tag each line of
% affiliation/address with \affaddr, and tag the
% e-mail address with \email.
%
% 1st. author
\alignauthor
Adrian Schr{\"o}ter\\
       \affaddr{University of Victoria, Canada}\\
       \email{schadr@acm.org}
% 2nd. author
\alignauthor
Jorge Aranda\\
       \affaddr{University of Victoria, Canada}\\
       \email{jaranda@uvic.ca}
\alignauthor
Daniela Damian\\
       \affaddr{University of Victoria, Canada}\\
       \email{danielad@cs.uvic.ca}
%% 3rd. author
%\alignauthor Lars Th{\o}rv{\"a}ld\titlenote{This author is the
%one who did all the really hard work.}\\
%       \affaddr{The Th{\o}rv{\"a}ld Group}\\
%       \affaddr{1 Th{\o}rv{\"a}ld Circle}\\
%       \affaddr{Hekla, Iceland}\\
%       \email{larst@affiliation.org}
}
% There's nothing stopping you putting the seventh, eighth, etc.
% author on the opening page (as the 'third row') but we ask,
% for aesthetic reasons that you place these 'additional authors'
% in the \additional authors block, viz.
\date{30 July 1999}
% Just remember to make sure that the TOTAL number of authors
% is the number that will appear on the first page PLUS the
% number that will appear in the \additionalauthors section.

\maketitle
\begin{abstract}

Building tools to help software developers communicate effectively requires a deep understanding of their communication dynamics. To date we do not have good comprehension of why developers talk to each other as a result of \emph{some} events in the life of their projects, and not of others. This lack of knowledge makes it difficult to design useful communication models and support systems.

In this paper, we narrow down the study of communication behaviour to focus on interactions that occur as a result of a particular kind of project event: the submission of a changeset to the project repository. In a case study with the IBM\textsuperscript{\textregistered} Rational\textsuperscript{\textregistered} Team Concert\texttrademark\ development team we investigate which factors influence developers to request information about a changeset to their product. We identify several such factors, including the development mode in which the team is operating, the background and recent performance of the author of the changeset, and the risk that the changeset poses to the stability of the product. Incorporating these factors into communication support systems and failure prediction models may lead to improvements in their performance.
\end{abstract}

% A category with the (minimum) three required fields
%\category{H.4}{Information Systems Applications}{Miscellaneous}
%A category including the fourth, optional field follows...
\category{D.2.9}{Software Engineering}{Management}[Programming\linebreak teams]
\category{K.6.3}{Management of Computing and Information Systems}{Software Management}[Software development]

\terms{Human Factors,Management}

\keywords{Information Sharing, Communication, Software Teams}

\section{Introduction}

% Setting the context
For all but the most trivial programs, software development today is a collaborative activity. The members of software teams must coordinate and communicate, often intensely, if their projects are to reach a satisfactory conclusion.
Previous research has shown that communication among developers has a profound influence on several aspects of software development~\cite{hinds:cscw:2006,wolf:icse:2009}.

%...but beyond a superficial account of communication gaps, research so far tells us little that we can use.
Despite this acknowledged importance of communication in software teams, our understanding of communication behaviour is still very limited. Ko \emph{et al.}~\cite{ko:icse:2007} point out that developers often require information about the work output of their peers.  We know little about why developers act upon this requirement. What causes developers to seek information in some cases, but not in others? There are a multitude of potential factors: the closeness of their peers' work with their own, the expertise of the developers, their physical proximity, their level of workload, and even, perhaps, the time of day. By identifying which factors actually influence the likelihood that a developer will request information about the events surrounding her, we might better understand the nature of software-centric communication, leading to better practical applications for our research.

For this study we decided to investigate communication revolving around changesets, since a study of changesets can be generic enough to be applicable in many settings, but particular enough to be informative about detailed patterns of behaviour. We report on a field study that used a mixed-methods approach (participant-observation, surveys, and interviews) to understand what technical, social, and process-based factors influence whether a developer will seek information about a changeset or not. We investigated communication in the IBM Rational Team Concert (RTC) project, which is distributed across North America, Europe, and Asia, and in which communication is linked to changeset data. The RTC project presents unique opportunities for research on communication in software teams not only because it takes place within a development environment in which a significant fraction of communication is recorded in technical artifacts, but also because the team is tasked with the goal of developing these artifacts themselves. The challenge of effective team communication is particularly present in the RTC project team. 

% JA: We can always just take out the next para if we need space. It's usually meh.
In the remainder of the paper we present the motivation and background for our study, and then describe our study design.  We present in detail the results of our analysis of the data, and distill a number of technical, process and social factors that we identified as influencing developers' information-seeking behaviour. We conclude by discussing the implications of our findings on the design of collaborative systems. 


\section{Background and Motivation}
\label{sec:rq}

% Introduce IBM RTC and the early works of SEGAL with it.
This research builds upon the growing body of work that studies coordination and communication in software organizations, which have long been recognized as playing central roles in leading software projects to success, e.g.~\cite{kraut:cacm:1995,curtis:acm:1988}.  

Our research group gained the opportunity to study the IBM Rational Team Concert (RTC) development team and the software artifacts produced by this team over several years. The RTC team maintains and cross-links all of their software and process artifacts using the very tool they produce. Partly due to its distributed nature, and partly due to its open source origins, the team's emphasis on written and archivable electronic communication results in a wealth of data for analysis.

Our previous investigations of the RTC project repositories revealed a significant relationship between gaps in communication structures and build success~\cite{wolf:icse:2009}. Therefore we proposed a recommendation system that allows team members to learn about who they should talk with in order to avoid build failures~\cite{schroeter:rsse:2008}. This recommendation system should focus on changesets and analyze significant differences between code dependencies and team communication structure to identify anti-patterns that the developers should attempt to break.
The basis for this principle is derived from Conway's Law~\cite{conway:datamination:1968} and by the model of socio-technical congruence proposed by Cataldo \emph{et al.}~\cite{cataldo:cscw:2006,cataldo:esem:2008}.
The model compares the communication network to the code dependency network and derives insights with respect to productivity from the gaps between those networks.

In order to build a good recommendation system, however, one needs to go beyond providing valuable information. As Murphy and Murphy-Hill show~\cite{murphy:rsse:2010}, we need developers to trust the recommender system: delivering the right recommendation at the right time is one important way to build such trust. Unfortunately, recommender systems often fall short on this front, saturating the developer with recommendations they neither need nor want, or presenting recommendations at inappropriate times. It is therefore important to know when developers actually need information, and what factors influence them to seek information. There are many potential factors that could change the likelihood that developers would want to obtain further information, and we do not have a sense, other than our intuition, to learn which factors are actually important and which are not.

We are not the first to study factors that influence communication among software developers. For instance, while APIs are intended to reduce coordination complexity among development teams, they also introduce barriers that hinder communication between teams by obscuring dependencies~\cite{desouza:cscw:2004,desouza:fse:2004}.
Software architecture, organizational structure and project age are also factors that influence communication and the awareness that developers have of each other's work~\cite{cleidson:tse:2011}. 
Processes, in particular, are often used to influence the communication among team members to reduce information overload~\cite{fussell:cscw:1998}, and geographic distance has been shown to affect communication, but its effects vary between projects~\cite{herbsleb:icse:2001,wolf:spip:2008}.
Nakakoji \emph{et al.}~\cite{nakakoji:code:2010} distilled the current knowledge about communication among developers into a set of guidelines to design tools and processes that facilitate effective expertise communication.

Those studies, and to our knowledge, literature in software engineering in general, only address communication on a project-wide level. However, to fully understand, and thus offer actionable knowledge to developers, we need to know what influences an individual developer's information-seeking behaviour.
We surveyed literature to compile a list of possible information-seeking triggers spanning change-related (such as code metrics), developer-related (such as experience level) and process-related (such as peer reviews) factors.

Therefore we pursue to answer the following \textbf{research question:}
 \emph{What factors influence whether a developer will decide to seek information about a changeset in her project?}


% \newpage%\ \newpage
\section{Study Design}
\label{sec:studydesign}

%We studied the communication behaviour around changesets of the software development team of the RTC project.
To address our research question, we complement our previous analysis of the RTC electronic repositories with rich qualitative data from interviews and observations, considering that much information on a project's lifecycle is not recorded in its repositories~\cite{aranda:icse:2009}. We focused on studying what factors cause a developer to seek information about a changeset, since changesets are the smallest units of change to a project that directly impact other developers.
%as well as having the potential to destabilize a project by introducing further failures.

Our research question called for a mixed methods empirical study. Therefore, we collected data about the team's information-seeking behaviour from three different sources:

\begin{enumerate}
\item The first author of this paper (Adrian) was embedded in a subteam of the RTC team as a participant-observer for four months.
\item We deployed a survey with the entire development team to validate some of the findings from the observations.
\item We conducted interviews with the developers from one component team to obtain richer information about their communication behaviour.
\end{enumerate}

\vspace{4mm}
\subsection{Study Setting}
\label{sec:rtc}
 
%(MORE HERE FROM PREVIOUS DESCRIPTIONS OF JAZZ IN OUR PUBS) 
Rational Team Concert is a scalable and extensible team collaboration platform that integrates source code management, issue tracking, and planning tools. The RTC platform uses a client-server architecture, with the server acting as a central data repository.
The team also uses RTC to develop RTC, and in this way RTC is a self-hosting project.
This helps the RTC team become aware of issues with the product before customers.% ANYTHING WE CAN SAY ABOUT ITS CUSTOMER BASE? 
The project follows a large scale iterative development process that emphasizes transparency~\cite{frost:ieeesoftware:2007}.
It defines six-week iteration cycles, which are  separated into planning, development and stabilization activities. A project management committee formulates the goals and features for each release at the beginning of each iteration, and the output of this process is assigned to the teams in the form of work items. The work items and nearly all documentation are publicly available.  %(except for the code repository)


The RTC development team has over 100 developers distributed across multiple sites in North America, Europe and Asia.  The RTC development team is subdivided into several teams that manage a component each, such as the source code system or the web user interface. To communicate with each other, especially across different geographical locations, the developers extensively use RTC's own issue and task management system. A particular characteristic of the RTC development environment---and one that made this study of communication behaviour around changesets possible---are explicit traceable links between work tasks, associated changesets, and developers' online comments about these tasks and changesets\footnote{Changesets in RTC are created in a developer workspace; each changeset consists of several changes to code files.}.

\vspace{4mm}
\subsection{Data Collection}

We used a mixed methods approach to answer our research question. We obtained data from three sources: participant observation, a survey, and a set of interviews with RTC team members.

\subsubsection{Participant-Observation}

The first author of this paper (Adrian) joined one of the RTC development sites in Europe for a four-month period in the Fall of 2010. He was involved as an intern with this development team, and thus was in a position to directly participate in the project development and communication activities. Our data collection opportunities were thus much richer than in typical observations conducted over shorter periods of time, or that do not involve active participation in the project.  During his period as a participant-observer, Adrian kept a daily activity log, recording any information of relevance to the communication behaviour of his team.

The observation period coincided with the months prior to a major release and during which the team focused on extensive testing rather than new feature development. As such, the majority of our observations are concerned with activities around testing the integration of RTC with other IBM Rational products, and it offered the experience of collaborating with many developers from different teams and across the remote sites. Finally, in the one month following the release, Adrian also participated in the development of a feature for the upcoming version of the product.

The team in which he was embedded had the responsibility to develop the task management and the planning components of RTC. There were ten developers at the site, including two novices that had recently joined the team and eight experienced developers. The team includes RTC's three senior project management members, who play a major role in planning the product's architecture and development.


\subsubsection{Interviews}

We conducted interviews with ten developers of the RTC team to inform our observational findings. The setting for the interviews was slightly unusual, and beneficial from a research perspective, since the first author's participant-observation allowed him to develop working relationships during his stay at the team, and to delve into a discussion of complex communication issues without needing to spend time understanding the basic context of the team.

In the interviews, we requested developers to provide details into their communication dynamics through a narration of ``war stories'' that directly related to communication among developers. ``\emph{War stories}'' are events that left an impression on the respondent. War stories emphasize specific events that they witnessed or took part in~\cite{lutters:ist:2007}; their narrative is particularly powerful at uncovering the elements of the stories that are relevant to the respondents. The interviews lasted between thirty minutes and one hour, and they were conducted at the end of the observation period. 


% \newpage
\subsubsection{Survey}

To complement the insights we obtained from the observations and interviews, we deployed a survey with the entire RTC team at the end of our observation period. The survey was designed iteratively and piloted with a European team, and intended to collect input about which factors increase the likelihood of a developer requesting information about a changeset from the changeset author. The items we included in the survey were in the categories of code-related, developer-related and process-related items. Each category included 14 items, as shown in Table~\ref{tab:surveyfactors}. \emph{Code-change related} items describe the changeset delivered or the code that the changeset modifies or affects. \emph{Developer-related} items relate to
the developers that delivered a changeset, the developer requesting the
information, and their relation to each other. Finally, \emph{process-related} items relate to the development phase in effect when the changeset was delivered, and to items relating to process requirements or practices. For each of the three categories, the survey asked the developers to rank each item in the category according to how strongly it increases their likelihood of requesting information about a changeset. 

We initially formulated our list of items from an analysis of the current literature on coordination and communication, as well as from our own previous research. We then refined the list of items through piloting and discussions with the development team. An initial version of the survey included fifty-nine items; the refined list had forty-two. Besides reducing the number of items in our list, the pilot survey led us to add several process-related items, as well as to change the format of the question (the initial survey had a battery of questions with Likert-scale answers; the final survey asked respondents to rank each item in comparison to the rest in its own category).

We deployed the survey through a web form, and invited the entire distributed RTC development team to participate. We sent a reminder one month later to increase the response rate. We obtained 36 responses to the survey; approximately a 25\% response rate.


\begin{table*}[tbh!]
\vspace{1pt}
\centering
\subfloat[Code-change-related items and quotes]{
\centering
\begin{tabular}{@{}l@{\hspace{5pt}}l@{\hspace{-5pt}}r@{}}
\toprule
Survey Items & Interview Quotes (I)/Participant Observation Quotes (O) & Points\\
\midrule
Changed API &(O) \emph{``Adrian, why did you change that API?''} [The RTC team avoids changing API.] & 390\\            	
Don't know why code was changed &(O) \emph{``I don't see a reason why you changed that code, you sure you needed to?''}& 348\\ 
Affects frequently used features &(I) \emph{``Before I ok a fix [even to a main feature], I ask if there is a workaround.''} & 325\\ 
Complex code &--- & 289\\	%\emph{``''}
Introduced new functionality &--- & 260\\ %\emph{``''}
Is used by many other methods &(O) {\small\emph{``Why did you fix this part? The bug is not in this part of the code, it is used and tested a lot.''}}& 240\\ 
Your code was changed &(I) \emph{``I'd be very irritated if someone other than [...] would touch my code.''}& 232\\ 
Stable code was changed &(O) {\small\emph{``Why did you fix this part? The bug is not in this part of the code, it is used and tested a lot.''}}& 213\\
Change unlocks previously unused code &--- & 192\\ 
 A bug fix &--- & 162\\%\emph{``''}	
 A re-factoring &(O) \emph{``I separate a re-factoring from a fix so that people can ask me questions about the fix.''}& 152\\
 Frequently modified code was changed &(I) \emph{``I like to know where the `construction sites' of the project are.''}& 118\\ 
 Code is used by few other methods &--- & 94\\ 	
 Simple code was changed &--- & 58\\ 		
\bottomrule
\end{tabular}
\label{tab:sub-technical}
}\vspace{20pt}
\subfloat[Developer-related items and quotes]{
\centering
\begin{tabular}{@{}l@{\hspace{5pt}}l@{\hspace{-5pt}}r@{}}
\toprule
Survey Items & Interview Quotes (I)/Participant Observation Quotes (O) & Points\\
\midrule
Author is inexperienced &(O) \emph{``Adrian, please put me always as a reviewer on you change-sets.''}& 385\\
Author recently delivered sub-standard work&(I) \emph{``She just changed teams and still needs to get used to this component.''}& 335\\
Author is not up to date with recent events &(I) \emph{``After you return from vacation we ensure you follow new decisions.''}& 329\\ 
You do not know the change-set author &(I) \emph{``I'd be very irritated if someone other than [...] would touch my code.''}& 269\\	
Author is currently working with you &(O) \emph{``We sometimes discuss changes we made to brag about a cool hack.''}& 226\\	
Author part of same feature team &--- & 190\\ 	
Author part of your team &--- & 188\\ 		
Worked with author before &--- & 165\\ 			
Busyness of yourself &(I) \emph{``If I see a problematic change-set, I'll ask for clarifications even if I'm busy.''}& 164\\
Busyness of the author &(I) {\small\emph{``If I need to know why an author made that change I just contact him.'' [Even if he is busy.]}}& 147\\
 Met in person &(I) \emph{``I've worked with him for 5 years now but never even met him.''}& 117\\
 Physical location &(O) \emph{``Here, America or Asia, I don't care, I ping them when they are online.''}& 112\\ 
 Author is experienced &--- & 93\\ %\emph{``''}		
 Author recently delivered high-quality work&--- & 86\\%\emph{``''}
\bottomrule
\end{tabular}
\label{tab:sub-social}
}\vspace{20pt}
\subfloat[Process-related items and quotes]{
\centering
\begin{tabular}{@{}l@{\hspace{5pt}}l@{}r@{}}
\toprule
Survey Items & Interview Quotes (I)/Participant Observation Quotes (O) & Points\\
\midrule
Release endgame & (O) \emph{``Adrian, in the endgame we only do minimal changes. Is your change minimal?''}& 319\\
To review the change &(I) \emph{``I often lack sufficient understanding of the part of the code I'm reviewing.''}& 311\\
To approve the change &--- & 298\\
Late milestone &--- & 293\\
During iteration endgame &(O) \emph{``Adrian, in the endgame we only do minimal changes. Is your change minimal?''}& 288\\
Obtain a review for the change &(O) \emph{``It is demotivating to get a reject, thus I talk to my reviewer beforehand.''}& 287\\
Obtain an approval for the change &(O) \emph{``I already fixed it, but I  need to convince my team lead to give me approval.''}& 281\\
Related work item has high severity &--- & 246\\
Verify a fix &(I) \emph{``Often I need to ask how I can tell that a change-set actually fixed the bug.''}& 235\\
Topic of work item the change is attached to &--- & 181\\
Priority set by your team lead&--- & 152\\
Role of the committer (e.g. developer) &--- & 145\\
Early in an iteration &(I) \emph{``... there are weeks I sometimes don't talk to my colleagues at all.'''}& 66\\
 Early milestone &--- & 60\\%\emph{``''}
\bottomrule
\end{tabular}
\label{tab:sub-process}
}
\caption{This table (in three parts) summarizes the data we collected from our three data sources: 1.) our participant-observer's log-book; 2.) survey responses from 36 developers situated around the world; 3.) interviews with 10 developers.  These items are ranked by points, which represent the cumulative importance the respondents assigned to each item in the survey.  Whenever possible, we provided quotes from interviews (I), or quotes from the participant-observation period (O) to exemplify the items.}
\addtocounter{table}{-1}
\label{tab:surveyfactors}
\end{table*}


%
%Please rank the triggers according to which is more likely to trigger you requesting information about a changeset.}. Survey rankings by the 36 respondents. The rank is determined by the points which is the sum of the reverse rankings provided by the developers.

\subsection{Analysis} 

For our data analysis, we transcribed the interviews, and then coded both the interviews and the participant-observation notes. From the codes we created categories, which we cross-referenced to the survey rankings. We used an open coding approach, during which we assigned codes to summarize incidents in sentences and paragraphs of the transcribed interviews that relate to our research question.

We calculated the ranking of the survey items for each category with a points-based system in which we added the reverse ranks given by each developer.
For instance, if a developer ranks an item to be first in its category, then it is worth 14 points (each category has 14 items).
In case of changes to the API, the first item in Table~\ref{tab:sub-technical} received 390 points which is the sum of nine developers ranking it first (9 $\times$ 14 = 126 points), eleven developers ranking it second (143 points), two developer ranking it third (24 points), four developers ranking it fourth, four fifth, and one sixth, seventh and ninth each. Four developers did not include this item in their rankings.

During our analysis, we derived a set of factors affecting communication behaviour based on our list of survey items and qualitative data from our interviews and observations. Whenever possible, we triangulated our findings using all our data sources. We tried to ensure that all of our findings were supported by at least two data sources; none of the findings we report were challenged by any of our data sources. 

Table~\ref{tab:surveyfactors} summarizes the ranking that each item received from the entire set of individual responses to our survey. Next to each item, we indicate, whenever possible, quotes from our interviews and observations that corroborate the position of the item in the list.


%\newpage%\ \newpage
\section{Findings}
\label{sec:findings}

% Introduce the findings
The evidence collected from our survey, interviews, and participant observation allowed us to derive seven factors that affect communication behaviour in software organizations, which we outline below. In the remainder of this section, text in italics refers to survey items that can be found in Table \ref{tab:surveyfactors}. Text in italics and around quotation marks denotes interview fragments.

\begin{figure*}[tb]
\centering
\includegraphics[width=\textwidth]{figures/findingProcess}
\caption{The pattern of information-seeking interactions throughout several iterations of a release cycle. Every release cycle consists of a number of iterations; each iteration includes an endgame phase. Changeset-based interactions are more frequent during endgame phases and during the last iteration of the release cycle.}
\label{IterationsFig}
\end{figure*}

\pagebreak
\subsection{Development Mode}

% Introduce development modes
One of the strongest findings from our study was the effect of the development mode that the team is in on its communication behaviour. Developers and managers alike give much importance to the development mode they are in, namely (1) normal iteration, and (2) endgame.

% Normal iteration
The normal iteration mode mainly consists of work that can be planned by the developer. This work tends to be new feature development, or modifications to existing features. Most of the planning for it has been laid out in advance; furthermore, each developer individually knows what features have been assigned to her, and can plan ahead to meet her obligations.
% Endgame
In contrast, the endgame mode mainly consists of work that is coming uncontrollably in short intervals from others. As a result of integration and more intense testing, defects crop up more rapidly and need to be addressed more quickly. Beyond allocating time for endgame activities, there is very little in this development mode that can be planned in advance.

% Cyclical pattern
The RTC team switches between the two modes in each iteration. Of the six weeks of a regular iteration, four weeks are assigned to normal iteration mode and two weeks to endgame. However, as deadlines approach, there are occasional special iterations that consist mostly of endgame-like work. In this manner, the same detailed pattern of alternation between normal iteration and endgame within an iteration is reproduced at a higher level, as shown in Figure \ref{IterationsFig}.

% Make it clear what the finding is
We identified the same pattern with respect to the amount of changeset-related interactions in the team. We found that the mode in which the team is currently in is an important factor determining whether developers will feel a need to communicate about changesets. Being in endgame mode increases the need for developers to communicate about a changeset, whereas being in normal mode decreases the need to request information about it.

% Participant-observer data
The first author of this paper (Adrian) experienced both modes during his participant-observer time with the RTC team. When he joined the team, it was in a release endgame phase, which was characterized by fixing bugs that were reported by testers. When the team released the project, it switched gears, and in the decompression after the endgame he had the opportunity to develop a feature for the product. The differences in the patterns of interaction between the two modes were distinctly clear, for him and for his peers. During the endgame mode, developers were essentially ``on demand,'' available to fix whatever bugs were discovered. Later, during the normal iteration mode, they experienced far more autonomy and control over their time, and began working on activities that were less demanding of an interaction back-and-forth, and especially less demanding of keeping track of the changesets committed by their peers. In our interviews, developers almost unanimously pointed to the separation of the two types of modes, describing the differences in the type of communication in each, and identifying the autonomous \emph{vs.}~uncontrollable, inside \emph{vs.}~outside influence.

% Some interview data
This is not to say that developers do not communicate while they are in normal iteration  mode, but that the nature of their communication seems to be different. In normal iteration mode, developers communicate less about concrete changesets, but they communicate more about high level ideas: for instance, they raise questions about how a feature fits into the existing architecture or general tactics to implement a feature, about how much of it actually needs to be implemented or can be reused from other libraries, and so forth. And as a developer commented in one of our interviews, \emph{``during feature development iterations there are weeks I sometimes don't talk to my colleagues at all.''} 

In contrast, the endgame mode is characterized by bug reports and last minute feature requests---that is, by work coming into every developer's desk uncontrollably. changesets become first class concerns during this mode because each change threatens the stability of the product: developers need to evaluate whether the changeset actually improves the product instead of making things worse. Therefore, they develop a habit of reviewing each other's changesets to assess the risk they pose to the project's stability.

% Survey data
Our survey data provides significant support for this finding. In Table~\ref{tab:sub-process}, we can see that overall, for the RTC development team at large, there is a greater likelihood to request information during product-stabilizing phases, such as 
\emph{during the release endgame}, and a lesser likelihood to request information when working towards an \emph{early milestone}. The former was ranked as the most important of the process-related items in our survey; the latter was ranked as the least important. All the survey items that refer to a late stage in the iteration or in the project lifecycle were ranked highly, and all the items that refer to an early stage were ranked lowly. This finding resonates especially with the team we interviewed. 

\begin{note}
The likelihood that a developer will request information about a changeset increases when the team is in a development mode characterized by uncontrollable work coming from outside.
\end{note}

%- number 1, 4, 5, 13,14
%- interviews: content changes
%-- sometimes frequency changes
%-- income vs out going
%-- other define vs self defined


\subsection{Perceived Knowledge of the changeset Author}

% Introduce this as another one of our Big Deal findings
One key determinant for a developer to seek information about a changeset consists of what the developer knows about the background of its author. Several aspects of the author's background seem to be important: the quality of her recent work, her level of experience, and her awareness of recent team events or decisions.

% Sub-standard work
First, these factors are important in part because they point to potential problems with the stability of the product. If a changeset author has \emph{recently delivered sub-standard work}, other developers will want to ensure that the new changeset will not deteriorate their product, and therefore they will try to find more information about it.

% Experience
Similarly, when the author is \emph{inexperienced}, or is not \emph{up to date} with recent team events for any reason, the risk of introducing problems into the product increases. One developer that interacts frequently with a newly founded team told us that \emph{``most of the work that I need to review is of very low quality and needs several iterations before it is up to our standards.''} Novices, generally speaking, face a ramp-up problem that takes a significant amount of time to overcome~\cite{begel:sigcse:2008}. This was the case with Adrian's work, too: his first changesets were subjected to more scrutiny due to his unfamiliarity with the code base.

% Survey data
The survey data backs up these observations. The three top items in Table \ref{tab:sub-social} consist of factors related to an unfavourable perception of the background of the changeset author. In comparison, the two least important items in that table refer to favourable perceptions of said background.

% It's partly about perception and trust
In essence, these items point to the relevance of trust in software development teams. For developers, trust in the skills of their colleagues is important. If this trust is non-existent (for instance, if a developer does not know the changeset author), developers are more likely to check the changeset and request information to ensure that the modifications were truly necessary and appropriate. But when trust exists (that is, when the author has a good record of delivering high-quality work, or is known to be an experienced team member), the need to seek information about a changeset decreases. 

\begin{note}
It is more likely that a developer will request information about a changeset delivered from an author that recently delivered sub-standard work, or is perceived as inexperienced or unaware of team events and decisions.
\end{note}


% observation
% no observations
%- experience 1
%- quality of recent work 2
%- up to date 3
%- don't know the author 4
%- experience 13
%- quality of recent work 14
%- role of the committer 12 (process)


\subsection{Common Experience and Location}

Shared work experiences affect the likelihood of information-seeking behaviour in the RTC distributed team. This is partly related to the trust and perception issues discussed above, but also to the establishment of interpersonal relations and to the intertwining work responsibilities and expectations.

According to our interviews, this is the case particularly for senior team members. Although they work in a globally distributed team, they have managed to establish personal relationships with developers at many different sites. Through continuous interaction and through social events, they have learned more about each other, and thus feel more comfortable to initiate contact with them. Junior team members, in contrast, know very few of their teammates globally, and since they simply do not have these interpersonal connections with the rest of their team, it is less important to them whether they know the author of a changeset when needing to contact them.

Generally speaking, despite the lack of opportunities to build personal relationships with off-site people, for most developers sharing common experiences makes it easier to contact teammates to request information. One team lead told us: \emph{``I have one or two contact people in each team we usually work with whom I ask for information and then often enough get referred to the right person in their team.''}

Our participant-observation data confirm this. At one point, Adrian needed to set up and test a specific component in the RTC product. The development team in charge of that component was located on a different site. However, since Adrian had previously had the opportunity to establish relationships with some of the developers of that team through a previous research study, it was easier for him to contact them and ask for help in his setup task.

Our survey data corroborate these observations. It suggests that the developers that form relationships (on a personal or work level) with other developers gave more emphasis on previous experiences with them. For instance, it made a difference whether the developers \emph{are currently working together} (Table~\ref{tab:sub-social}). Additionally, whether they have shared work experiences seems to be a more determinant factor than having \emph{met in person} or sharing the same \emph{physical location}.

This last finding would seem to indicate that the RTC team has managed to overcome obstacles created by geographical distance. Developers state that the physical location of a changeset author is not an important factor to seek more information about the changeset. Nevertheless, we should note two things. First, this finding merely points out that developers have no greater need to inform themselves of work output of their remote peers than of their local peers; it says nothing about the ease with which such information is acquired. Second, the RTC team has determined to carry out as many of its interactions as possible through textual, electronic media. This neglects the natural advantages provided by proximity in favour of uniform accessibility of interactions, no matter the physical location of the interlocutors.

\begin{note}
Recent or shared work experiences eases the act of requesting information from distributed or local developers.
\end{note}

%- currently working with author 5
%- same feature team 6
%- same team 7
%- worked with before 8
%- met author in person 11
%-- physical location 12


\subsection{Risk Assessment}

A central factor at the heart of the decision to seek information is a concern with the quality of the product and components developed by the team at large. Both managers and developers share this concern. Because of it, every team member is constantly evaluating whether there are significant risks involved in accepting a changeset and including it into the final product. This concern is greatest close to releases or major milestones, and less important when the team is in normal iteration mode.

Developers request information more frequently about changesets that touch on code that has a high customer impact. According to our interviews and observations, code parts with a high customer impact are those that are directly related to frequently used features or changes to the API that might be used by customers to customize the RTC product. In the case of the impact of API changes, there is an extensive knowledge exchange in the jazz.net forums between customers and developers about how to use the API to build extensions to RTC, which gives developers plenty of information about the possible impact of API changes to the customers. Consequently, changesets that modify code that has less impact on the customer are of less concern and less likely to be discussed. For example, when scanning fixes to reported bugs, developers perform a risk assessment to determine whether the bug fix is even needed. In the words of one team member, \emph{``with every tenth bug fix you introduce another bug to the system, so unless the bug does not have a workaround and is something most customers would experience, we give it low priority.''}

The ranking of items in Table \ref{tab:sub-technical} provides additional insights into the kinds of changesets that carry a risk for developers. A \emph{changed API} comes at the top of the list, code that \emph{affects frequently used features} comes third, immediately followed by changes to \emph{complex code}, changes that \emph{introduced new functionality}, and changes to code that \emph{is used by many other methods}.

\begin{note}
Developers investigate more carefully those changesets that might have a high impact to the customers or to the stability of the code base, and thus they are more likely to request information about them.
\end{note}

%- API 1
%- don't know why 2
%- frequent features 3 (customer)
%- complex code 4
%- new functionality 5
%- used by many others 6

%- stable code was changed 8
%- unlock previously unused code 9

%- frequently code was changed 12
%- code is used by few others 13
%- simple code was changed 14


\subsection{Work Allocation and Peer Reviews}

In the later stages of the development cycle, developers have two process-based constraints that prompt them to interact with their peers. First, they need their peers to review their changesets before they are included in the product. Second, they need an approval to start working on a task (a triaging process needs to take place so that developers are allocated to the most important work items).

In the changeset review process, there are two ways the changeset author and the reviewers interact.
Either the changeset author submits his changeset for review, or the changeset author discusses the changeset with the reviewer before submitting it for review. The main discriminant between meeting with the reviewer to discuss the changeset before submitting it or submitting without prior discussion seems to be whether the reviewer and the author are co-located.

Adrian's mentor during his time with the RTC team explained: \emph{``It is very demotivating to get a changeset rejected, that's why we usually try to discuss thing before and then the reviewing just becomes all about quickly testing and accepting the changeset.''} Of course, this does not prevent an already discussed changeset from being rejected, but it signals that the process-based constraint is not as strict as to forbid discussions about a changeset between authors and reviewers in the interest of an unbiased code review.

In the case of developers seeking approval to start working on a task, the process is similarly flexible. The approval is supposed to be issued before the developer begins work on the task. However, we observed several times that developers performed quick fixes, and later went to their team leads to acquire approval for the fixes they had already performed.

Several items in our survey point to the importance of information-seeking around the processes of approval and peer review. In Table \ref{tab:sub-process}, the second- and third-highest items were information needed because the respondent needed \emph{to review the change} or \emph{to approve the change}. The need to  \emph{obtain a review for the change} or to \emph{obtain an approval for the change} were of moderate importance. \emph{Verifying a fix} was not judged as important by our respondents.

\begin{note}
Peer review processes (needing to review or to have one's work reviewed) and approval processes (requiring an approval or granting an approval) trigger people to discuss changesets.
\end{note}

%- need to review 2
%- need to approve 3
%- need a review 6
%- need an approval 7
%- need to verify fix 9 

\subsection{Business Goals vs Developers' Pride}

Senior team members (managers and team leads among them) often ask about the purpose and the real need of a changeset in order to minimize unpleasant surprises that might be introduced with further code changes, and simply to have an economically feasible product. One senior development lead told us: \emph{``often I need to stop my developers from fixing every little bug otherwise we would never be able to ship.''}

Although every team member is concerned with the quality of their product, a more realistic, business-oriented perspective seems to be more prominent in the senior team members' information-seeking behaviour: they may request clarifications on the business case of changesets or work items if they are not immediately clear in the existing documentation.

More junior team members, however, do not have this concern. They express their pride in the actual product and want it to be as bug free as possible. A developer commented to us: \emph{``I want to be proud of what I deliver and I am fairly certain if we don't fix it now it will come back to haunt us because it might upset some customers.''} Not only does this lead the developer to try to convince her manager that including changesets for seemingly unimportant fixes is valuable, it also makes her information-seeking behaviour different with respect to that of the more senior team members. Broadly speaking, managers will be more interested in seeking clarifications as to the business case of a work item if it is not clear in its description, while developers will not care as much about this.

\begin{note}
Senior team members discuss changesets when their importance to the product's release is not evident in their archives.
\end{note}


\subsection{Type of Change}

The three main reasons for a developer to commit a changeset are to (1) do feature development, (2) fix bugs, and (3) re-factor. Among them, feature development seems to be the one that prompts developers to seek information the most.
In our survey (Table \ref{tab:sub-process}), code that \emph{introduced new functionality} is ranked fifth among the code-change related factors, while changes that consist of \emph{a bug fix} and \emph{a refactoring} ranked far lower on the same list.

This may be due to the fact that feature development has less tool support than bug fixing (which is supported by debugging tools and automated test frameworks) and refactoring (which is essentially an automatic process today). The limited tool support and the inherently difficult nature of the task itself makes changesets that introduce new functionality more difficult to understand or assess. Hence, developers are more likely to request information about them.

\begin{note}
Among all types of changesets, those that revolve around feature development are most likely to make developers request information about them.
\end{note}


%- change introduces new functionality 5
%- change was bug fix 10
%- change was re-factoring 11


%\subsection{Code Ownership}
%\newpage

%\begin{note}
%...
%\end{note}
%\newpage

%The RTC software itself is organized in different components with each component owned by a team.
%Each component is subdivided into several modules and each team member takes care of multiple modules.

%% survey
%%According to the survey

%% interview
%The interviewees who commented on code ownership explained that they do not generally oppose to others changing methods they own.
%But they prefer that people that change the code they own are people they know and know that they have sufficient expertise with the code they change.
%One developer that works with the India team said that it wouldn't be a problem when one of his co-located team mates would change her code but he would be very concerned if a developer from the less experienced India team changed her code.

%% observation
%\newpage
%- your code was changed 7


%\newpage\ \newpage
%\pagebreak
\section{Discussion}
\label{sec:implications}

Previous research has made it clear that developers communicate frequently to satisfy a number of information needs. However, in order to help them communicate and perform more effectively, we need better models of their communication behaviour. More specifically, we know that developers need to stay abreast of the work committed by their peers, but we do not know which, among the many pieces of work their peers commit, they actually care to know more about, and which do they feel they can safely ignore.

In this study, we provide some answers to this question by uncovering a number of factors that influence the information-seeking behaviour of developers. Our point of origin was the fact that developers often need to learn more information about the changesets committed by their colleagues. We aimed to discover the factors that trigger them to look for further information when they do.

%Development Mode
One major factor that influences developers' information-seeking behaviour is the development mode in which their team is in. Developers differentiate between two main development modes: normal iteration mode and endgame mode. These modes differ deeply by the kind of discussion that developers have about their products, and by the amount of control and autonomy they have over their activities. The tools and heuristics that work for one are not likely to work for the other. Fortunately for researchers and tool designers, at least for the RTC team it is  usually clear which mode the team is on, as the transition between normal iteration and endgame modes is made explicitly. Therefore, this is a finding that can be exploited with relative ease in models and tools that support developer information-seeking behaviour.

%Type of Change; work approval; peer reviews
Other findings from our study are also directly applicable in practice. Communication behaviour is at least partly determined by the organizational process, such as requests to perform code reviews and requests for approval to address certain work items during endgame; these communication patterns can be understood and modelled. Additionally, developers consistently reported that they request information about certain tasks more frequently than others. They tend to be less likely to ask about routine re-factorings or bug fixes than about changes introducing new features or functionality. Bug fixes and re-factorings do not change the expected behaviour of a product, while new functionality introduces risk to the product; therefore it requires additional scrutiny.

%Risk Assessment
Risk assessment, in fact, guides many developer decisions on whether to engage or interact with others to discuss changes made to the product. For example, changesets that affect many parts of the code (such as changes to the API) raise flags more often than changes which are fairly isolated. But there are two kinds of risks---the immediate risk that the product contains failures, and the long-term risk that a certain change increases maintenance in the future or takes the product in the wrong direction.
%Manager Process vs Developer Pride
This balance is hard to strike. Broadly, team members in junior positions seem concerned with the former, and those in more managerial positions (senior developers and team leads) with the latter; their information-seeking patterns vary accordingly.

% Trust issues
Trust arose as a major issue throughout our findings. The changeset author's background affects whether a developer will seek further information on the change itself: if the developer trusts the author (that is, if they perceive that the author has the skills necessary to implement the relevant changes with little assistance), she will feel a lesser need to oversee their work. At the same time, however, if two developers have shared work experiences, the threshold to establish communication between them decreases. In effect, this means that two developers may communicate more often about changesets even if none of them authored them: for instance, one developer could be acting as a bridge or a connector between the other developer and the right person to discuss a changeset in a different team. This is a practice commonly found in distributed teams \cite{wolf:spip:2008}.

\subsection{Recommender System Design Guidelines}
\label{sec:sub:tools}
Nakakoji \emph{et al.}~\cite{nakakoji:code:2010} formulated nine design guidelines for systems supporting information-seeking in software teams. Some of them deal with minimizing the interruptions experienced by the developers who are asked for information, while others refer to enabling the information-seeker to contact the right people. Our findings help us refine two guidelines from Nakakoji \emph{et al.}~that specifically refer to integrating the expertise communication with developer activities and to contextualizing the information it provides. We offer the following four design guidelines:


%adjust to development phase
\textbf{Guideline \#1}: \emph{Communication support systems should adjust to the development mode.}
The information-seeking behaviour of developers changes drastically with their development mode. When in normal iteration mode, developers act upon planned work and can therefore anticipate the information they need. Tools from which developers pull information, such as Codebook~\cite{begel:icse:2010}, support this kind of information-seeking behaviour, whereas tools that create a stream of information from events as they happen (like Ensemble~\cite{xiang:rsse:2008}) interfere with a developer working in normal iteration mode by constantly flooding them with information.
%
But in endgame mode, when developers react to unplanned incoming work, such as bug reports or requests for code reviews, notifications and additional information about relevant events greatly supports their tasks. In this case, Ensemble-like interventions become helpful, whereas obtaining information through tools such as Codebook introduces extra steps to obtain similarly useful information.

%Task type
\textbf{Guideline \#2}: \emph{Communication support systems should assist in non-implementation tasks such as code reviews and risk assessment.} 
There are many systems that support typical development tasks such as debugging and re-factoring. We found that those tasks have little influence on the information-seeking behaviour of developers. But reviewing or assessing the risk of a changeset has a profound impact on information-seeking behaviour. When developers review or assess the risk of a changeset, they need access to a considerable amount of low level information concerning the changeset, such as information about the motivation behind the changeset and about the code components affected by it. Thus, providing information to developers while they perform tasks such as code reviews or risk assessments supports their current information-seeking behaviour.


%Business goals
% JA: Not sure I really like this one...
\textbf{Guideline \#3}: \emph{Communication support systems should account for business goals.}
Software teams need to triage and focus the team's efforts on the work items with the highest priority. Some teams record such prioritization under the priority or severity fields of their work items, but these fields tend to have an emphasis on the technical challenge or stability of the product, not on the extent to which the fulfillment of the work item brings the organization closer to satisfy its business goals. Systems supporting the information-seeking behaviour of developers should be able to hide, upon request, the information related to tasks that are not mission critical to the organization, helping the team focus its attention on the most relevant problems.


%Adjust to cs-author perceived knowledge
% JA: Unsure about this one as well, as it stands.
\textbf{Guideline \#4}: \emph{Communication support systems should account for perceived knowledge of other developers.}
Several factors that trigger developers to seek information about a changeset are not related to the code in the changeset itself, but to its author. The author's experience level, as well as the quality of her previously delivered work, has a great influence on whether developers will seek information about their most recently submitted work. We suggest to provide a filtering mechanism that accounts for parameters such as author experience and performance.


\subsection{Explaining Failure Prediction}

Our findings have implications to the current insights in software failure prediction. The literature on failure prediction has identified factors and developed models that cause the software product to be more or less susceptible to failures. These models are often limited in providing explanatory power as to why those particular measures are good failure predictors. Our findings provide possible explanations on a number of powerful failure predictors.

We found that shared work experiences substantially increase the chance that developers will request information from each other. This implies that developers who have never worked together (an occurrence that is more common among members of different teams) are less likely to request information from each other. Lack of familiarity, as an inhibitor of information-seeking behaviour, thus helps explain why organizational distance is a strong failure predictor~\cite{nagappan:icse:2008}.

Similarly, we found that distance has an insignificant influence on whether they request information about a changeset. This helps explain why Bird \emph{et al.}~\cite{bird:icse:2009} could not find any relation between the physical distance among developers working on the same binary and the quality of the corresponding binary: developers have adapted to textual, electronic communication media, no matter whether their interlocutors are nearby or halfway around the world.

We observed that the amount of changeset-related information seeking spikes during the endgame of an iteration (see Figure \ref{IterationsFig}). This cyclical repetition of bursts in communication might be linked to the pattern observed by Kim et al.~\cite{kim:icse:2007} in relation to failures at changeset level. We hypothesize that these bursts could coincide with the development mode the team is in, so that incorporating knowledge about development modes might improve failure prediction models further.


%\newpage
\section{Threats to Validity}
\label{sec:threats}

Our survey asked the developers to rank a pre-defined list of items in
terms of the influence they may have on their information-seeking behaviour.
In doing so, we attempted to reach a compromise between the survey length and
survey response rate. 
This bias poses a threat to our findings due to the possibility that we were missing important items.
We mitigated it by developing the survey iteratively by piloting and discussing it with one of the development teams to identify the most important items, and by relying on our other two sources of data to triangulate our findings.

%While presenting developers with the pre-defined list of items may seem as
%limiting their ability to refer to all possibe relevant factors, our survey
%design attempted to minimize the potential risk of this approach. Our pilot
%survey included open-ended questions in which developers could suggest different
%factors. In analyzing the results of the pilot we were able to identify the
%most ranked factors to include in the final version of the survey. 


We had the chance to interview ten developers, which represent a fraction of the development team at large. These ten developers were all located at the same site. As a result of this, our interview data could be biased and unrepresentative of the RTC team at large.
However, we are confident that this threat is minor, due to the mix of developers we interviewed, including novices, senior developers, and team members that had been part of the group since its beginning.
Furthermore, triangulation with our observations and survey responses increases our confidence in our findings.

Our data come exclusively from the IBM RTC development team, and the extent to which our results are valid for other companies is not clear without replications. 
The RTC development team has more than one hundred members, it interacts with a number of customers, and it is in the process of integrating its product with others.
These are all forces that shape communication dynamics. 
However, their interaction patterns seem to be similar to those of other globally distributed teams that need to synchronize work across multiple time zones.

\section{Conclusion and Future Work}
\label{sec:conclusions}

Coordinating work among developers is a keystone of successfully shipping a software product~\cite{herbsleb:icse:2001,kraut:cacm:1995}. To fully understand how developers coordinate, it is necessary to understand the factors that influence their communication behaviour. Among our findings, we saw that the current phase in the development process and worker experience were two  major factors that influenced whether developers communicate about a change. Tools providing information and research on coordination in software engineering should consider and incorporate our list of factors in order to help developers interact more effectively.

Our long-term goal is to implement a system that recommends key developers to contact about problematic changesets. Looking back at the findings of this study, we can make two conclusions on this topic:
(1) there is value in implementing such a recommender system, since developers monitor and assess carefully the information related to changesets in their projects, especially to the extent to which it relates to software quality, and 
(2) the information-seeking behaviour of developers is affected by more than just the technical nature of a changeset. Individual and process-based information consistently informs their decisions to seek information about a changeset or not.


%\end{document}  % This is where a 'short' article might terminate

%ACKNOWLEDGMENTS are optional
\section{Acknowledgments}

We thank the members of the RTC team for their invaluable participation in our study. We also thank Julius Davies and Irwin Kwan for their comments and feedback on an earlier version of this paper.

%
% The following two commands are all you need in the
% initial runs of your .tex file to
% produce the bibliography for the citations in your paper.
\bibliographystyle{abbrv}
\bibliography{fse2011-schroeter}  % sigproc.bib is the name of the
%Bibliography in this case
% You must have a proper ".bib" file
%  and remember to run:
% latex bibtex latex latex
% to resolve all references
%
% ACM needs 'a single self-contained file'!
%
%APPENDICES are optional
%\balancecolumns

\end{document}
