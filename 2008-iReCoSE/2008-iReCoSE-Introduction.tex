\section{Introduction}
Accessing research data in Software Engineering is very difficult. Regardless of
the project methodology, the high cost of labour and developers' stressful
workload are always big challenges that researchers face in their effort ot
access developers' working environment. On the other hand, software developers
usually utilize digital media to coordinate their work. This media is 
usually archived and thus can be used for research as it provides valuable
insights into the developers' activities. For example, the bugzilla bug
tracking database of several open-source development teams have been studied in
the past~\cite{bettenburg:2007:tr}. So did developer mailing lists, emails~\cite{Bird:2006:MSR}, or even IRC chat~\cite{Cataldo:2006:CSCW}.


However, this media are usually not connected to each other. It is very difficult
for researchers to link artifacts such as a bug on bugzilla and a post on the
developer mailing list to make them useful for research. Fortunately, the advent
of integrated end-to-end collaboration development environtments such as IBM Jazz~\cite{IBM:2008:url} provides research with opportunities to explore the development activities without having to take away developers' valuable working hours or manually connecting different artifacts from different media. In
this paper, we are describing our experience in data mining the IBM Jazz team
collaboration. In the next section, we present our conceptualization of doing
research via artifacts as well as challenges we encountered. Then we explain the
importance of the IBM Jazz development environtment to research. In Section 4, we
explain the diffent options and contraints we have during the mining process.
Section 5 outlines the three design principles we learned from building the data
mining tool. And finally, in Section 6, we provide some results of a research
project we did based on the data we mined in Jazz.


