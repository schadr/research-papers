\section{Why Jazz}
Jazz~\cite{Cheng:2003:ACMQueue,frost:2007:IEEESoftware} is a development
environment that integrates programming, communication, and project management.
The main goal of Jazz is to bring collaboration support for all of the software
development activities such as planning, building, testing, and reporting. For
developer, this means that they can communicate and coordinate their work through
the very same tools they use to develop code. To achieve this, Jazz provides a
central repository for all tools supporting the development process such as a planning
wiki, a work item tracking system, a report component, and a build system.

At the time of our study, Jazz consisted of three major
components~\cite{IBM:2008:url}:
\begin{itemize}
\item The Jazz repository. 
\item The Rational Team Concert (RTC) client.
\item The dashboard, a web user interface.
\end{itemize}
The \emph{Jazz repository} stores all artifacts available to the two clients. 
Examples for such artifacts are work items, comments, and build results. 
The \emph{RTC client} is the main development client. 
It is built upon the Eclipse platform and allows developers to control their work spaces, to perform version control functions, and to request builds. 
The \emph{web user interface} allows developers to access most of the non-coding related artifacts, such as work items, iteration plans, and reports. 

For researchers, Jazz provides an easy way to extract valuable research data. Due
to the tight integration of the different components, links of artifacts to
related software development activities already exists. For example, a comment
is linked to tasks, also called work items, and source code changes in the form
of change sets are attached to these work items. This is a major advancement to
other systems that used their own heuristics to link software
artifacts~\cite{cubranic:2003:icse}.

%So if a build is red, we can find all of the communication around the build and, subsequently, construct the communication network associate with the red build.

%In the next section we explore the different options to extract Jazz data and the constraints we face during the process.

