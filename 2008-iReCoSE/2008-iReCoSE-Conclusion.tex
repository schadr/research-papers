\section{Conlusion}
Mining the Jazz repository has been a challenging journey. In
this paper we shared our experiences and the lessons we learned during the
process: iterative, minimal invasive, and separation. We think that the design principles we presented here
are applicable to any research via artifact projects regardless of the repository
behind it.

For us, the Jazz data continue to be a very valuable source of information. The
artifacts in Jazz hold valuable insights about both the social and technical
aspects of the Jazz team's development process. Obtaining the communication and
coordination data in one place creates numeorous of possibilities for researchers
to look at various aspects of intricate software development activies all
together, and which were very hard to do with research methods other than
repository mining. With the release of Jazz and the adoption of technology by
many of the Eclipse based products from IBM, we think that mining the Jazz
repositories will bring more and more benifits to research.
