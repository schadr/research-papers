\section{Design principles}
The main requirement for our tool besides being able to extract all data is to
not cause any problem in the development cycle of the team, such as causing to
much load on their repository while extracting the data. We implemented these
requirements in a program that was developed in several iterations.

\paragraph{Iterative} 
At the beginning of developing our mining tool, we evaluates the data that was
available through the RTC client API. We found that the API did not support the
extraction of all the data so we proceeded with an iterative development that was
oriented at the different kind of data accessible through the API at a time.
Similar to iterations in an Agile process, we extended the extraction capability
of the tool one by one via adding data query set. At the end of each iteration,
we asked our research contact at IBM to extract more data with the updated tool.
This had the advantage that we received additional feedback about API usage from
our IBM research contact.

\paragraph{Minimally Invasive} 
The data quantity we want to extract can cause a high load on the repository and
thus disrupt the development process, each time we extract the data.
% After several iteration, our query set started to put very high load on the
% main repository which prevented developers access to it.
We avoided causing such high repository load by updating the existing data
instead of extracting all data every time we run the program. This had the
advantage in case we finished a new query we did not need to extract all the data
we extracted earlier again.

\paragraph{Separation}
In order to keep the additional load minimal on the repository we tried to remove
redundancies in the queries as well as avoid the queries to load data that has
already been extracted. At the beginning, some of our queries depended on each
other or we could not avoid loading already extracted data. For example, if we
wanted to fetch the new comments on work items, we had to fetch the work items
themselves although they had not changed since the last extraction. This
introduced additional load on the repository. We removed all unnecessary
dependencies among queries and tried to minimize loading data that has already
been extracted.

% We think those decisions were important to the success of the project and may
% be useful for future research.
