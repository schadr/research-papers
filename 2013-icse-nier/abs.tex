\begin{abstract}
The misalignment between the social and technical dimensions of software development has been
linked to losses in developer productivity and increases the likelihood of a build to fail. Although research has produced empirical evidence of this relationship, it has yet to produce actionable knowledge that managers and developers can act upon to avoid build failure.
We propose an approach that generates recommendations to alleviate the effect that mismatches between the social and technical dimensions of a software project has on  builds failure. Specifically, our approach analyzes the technical dependencies, actual coordination and all builds in a project, and recommends a ranked list of developer pairs that, if present in the current build, will increase the current build's chance of failure. We discuss our preliminary evaluation of our approach using data from the IBM Rational Team Concert\texttrademark\ project and outline future steps in our research. 


%showed that we can recommend pairs of developers that form the mismatch between the social and technical dimensions.
%These pairs were statistically related to build failure and offer to managers and developer opportunities to improve the chance that the current build will succeed.
%Two developers that form a mismatch can improve the alignment between the social and technical dimensions by initiating coordination.
%
%
%
%In a case study of
%coordination in the IBM Jazz\texttrademark\ project, we investigate the
%communication and technical dependencies between developers involved in
%software builds and relate their misalignment to the build failure.  
%Through our proposed approach we identified a number of developer pairs that did not communicate  their dependencies and thus increased the likelihood of build failure. 
%Upon this actionable knowledge, developers and managers can act to prevent build failure. 
%If any one of these pairs is present in a build's social network, the build had at least a 74\% chance to fail. 
%This has several practical implications for the design of collaborative systems, such as the integration of recommendations about inter-personal relationships.
\end{abstract}