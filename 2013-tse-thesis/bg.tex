% !TEX root = thesis-journal.tex
\section{Background}
\label{chap:bg}
In this Section we provide an overview of related work: (1) the research on coordination in software development teams and (2) failure prediction using social networks.

\subsection{Coordination in Software Engineering Teams}
Software is extremely complex because of the sheer number of dependencies~\cite{sawyer2004:teams}.
Large software projects have a large number of components that interoperate with one another.
The difficulty arises when changes must be made to the software, because a change in one component of the software often requires changes in dependent components~\cite{desouza:2008}. Because a single person's knowledge of a system is specialized as well as limited, that person often is unable to make the appropriate modifications in dependent components when a component is changed.

Coordination is defined as ``integrating or linking together different parts of an organization to accomplish a collective set of tasks''~\cite{vandeven1976}. In order to manage changes and maintain quality, developers must coordinate, and in software development, coordination is largely achieved by communicating with people who depend on the work that you do \cite{kraut:1995coordination}.

Thus, a developer should coordinate closely with individuals whose technical dependencies affect his work in order to effectively build software. This brings forth the idea of aligning the technical structure and the social interactions \cite{herbsleb2007:fose}, leading us to the foundation of socio-technical congruence.

%Research in software-engineering coordination has examined interactions among
%software developers \cite{carter2004,marczak:re:2008}, how they acquire
%knowledge \cite{ehrlich:icgse:2006,nakakoji2010:rdc}, and
%how they cope with issues including geographical
%separation~\cite{espinosa2007:team_knowledge,herbsleb2003:speed}.
%The ability to coordinate has
%been shown as an influential factor in customer satisfaction \cite{kraut:1995coordination} and  improves the capability to produce quality work~\cite{faraj2000}.


Software developers spend much of their time
communicating~\cite{perry94} and research documents reasons for communication as well as challenges in developers' communication. Because developers face
problems when integrating different components from heterogeneous environments~\cite{redmiles2007:continuous},
developers engage in direct or indirect
communication, either to coordinate their activities, or to acquire knowledge of
a particular aspect of the software ~\cite{nakakoji2010:rdc}.While Ko et al~\cite{ko:icse:2007} found that developers were identified as the main source of knowledge about code issues, de Souza et al~\cite{desouza2007:awarenessnetwork} found that 
communication is important to avoid collaboration breakdowns and delays.
Herbsleb, et al examined the influence of coordination on integrating software
modules through interviews~\cite{herbsleb1999:architectures}, and found that
processes, as well as the willingness to communicate directly, helped teams
integrate software. While this evidence points to the role that developer communication plays in achieving successful software integrations, a first step in our investigation was to collect 
%Wolf et al~~\cite{wolf:icse:2009} used properties of social networks to predict the outcome of integrating the software parts within teams.
%This prior work establishes the fact that developers communicate heavily about technical matters.

However, coordinating software teams do face challenges that are greater as the distance between people increases \cite{herbsleb:icse:2001}.
Studies of Microsoft~\cite{bird2009:dds_quality,nagappan:icse:2008}
show that distance between people that work together on a
program determine the program's failure proneness.
Differences in time zones can affect the number of defects in software projects \cite{cataldo2009:quality}.

Although distance has been identified as a challenge, advances in collaborative
development environments are enabling people to overcome challenges of distance.
One study of early RTC development
shows that the task completion time is not as strongly affected by distance as in previous studies~\cite{Nguyen:2008Distance}. Technology that empowers distributed collaboration include topic recommendations~\cite{carter2004} and instant messaging~\cite{niinimaki2008}. Processes are adapting to the fast pace of software development: the Eclipse way~\cite{frost:ieeesoftware:2007} emphasizes placing milestones at fixed intervals and community involvement.
These new processes lie the Eclipse way that focus on frequent milestones lends more importance to software builds warranting more support by research as we conduct.

\subsection{Coordination and Failure}
\label{sec:ResearchQuestions}

%\paragraph{Communication and failure}
%Social network analysis has an extensive body of knowledge of analysis and implications with respect to communication and knowledge management
%processes~\cite{Burt:1995vo,Freeman:1979rl}. Griffin and
%Hauser~\cite{Griffin:1992ms} investigated social networks in manufacturing teams.
%They found that a higher connectivity between engineering and marketing increases
%the likelihood of a successful product. Similarly, Reagans and
%Zuckerman~\cite{RayReagans:2001os} related higher perceived outcomes to denser
%communication networks in a study of research and development teams.
%
%Communication structure in particular -- the topology of a communication network
%-- has been studied in relation to coordination
%(e.g.~\cite{hossain:cscw:2006,hinds:cscw:2006}) and a number of common measures of
%communication structure include network density, centrality and structural
%holes~\cite{Wasserman:1994sq,Freeman:1979rl}.
%
%Density, as a measure of the extent to which all members in a team are
%connected to one another, reflects the ability to distribute
%knowledge~\cite{Rulke:2000ys}. Density has been studied, for example, in relation
%to coordination ease~\cite{hinds:cscw:2006}, coordination
%capability~\cite{hossain:cscw:2006} and enhanced group
%identification~\cite{RayReagans:2001os}.
%
%Centrality measures indicate importance or prominence of actors in a
%social network. The most commonly used centrality measures include degree and
%betweenness centrality having different social implication. Centrality measures
%have been used to characterize and compare different communication networks
%constructed from email correspondence of W3C (WWW consortium) collaborating
%working groups developing new technical standards and architectures for the
%web~\cite{Gloor:2003cikm}. Similarly, Hossain et al~\cite{hossain:cscw:2006}
%explored the correlation between centrality in email-based communication networks
%and coordination, and found betweenness to be the best measure for coordination.
%Betweenness is a measure of the extent to which a team member is
%positioned on the shortest path in between other two members. People in between
%are considered to be ``actors in the middle'' and to have more ``interpersonal
%influence'' in the
%network(e.g.~\cite{Gloor:2003cikm,zimmermann:icse:2008,hossain:cscw:2006}).
%
%The structural holes measures are concerned with the degree to which there
%are missing links in between nodes and with the notion of redundancy in
%networks~\cite{Burt:1995vo}. At the node level, structural holes are gaps between
%nodes in a social network. At the network level, people on either side of the
%hole have access to different flows of information~\cite{Hargadon:1997asq},
%indicating that there is a diversity of information flow in the network.
%Structural holes have been used to measure social capital in relation to the
%performance of academic collaborators (e.g.~\cite{Brambila:PICMET2007}).
%
%Most prediction models in software engineering to date mainly leverage source
%code related data and focus on predicting failing software components or failure
%inducing changes
%(e.g.~\cite{bell:2005tse,schroeter:isese:2006,zimmermann:icse:2008,kim:2008tse}).
%And only few studies, such as Hassan and Zhang~\cite{hassan:ase:2006}, stepped away
%from predicting component failures and used statistical classifiers to predict
%integration outcome.
%We want to extend the body of knowledge surrounding prediction models using communication data or focusing on build outcome by investigating how to improve communication among software developers to prevent build failures.





%\subsection{Related Work}
%\label{sec:relwork}
%We aim to integrate work that investigates team collaboration to produce actionable knowledge upon which developers can act.
%Several studies are relevant with respect to different dimensions of our work:

\paragraph{Factors that affect software builds}
To the best of our knowledge the studies by Hassan et al.~\cite{hassan:ase:2006}
and Wolf et al.~\cite{wolf:icse:2009} are the only studies that conducted
research to predict build outcome. Hassan et al.~\cite{hassan:ase:2006} found
that a combination of social metrics (e.g. number of authors) and technical
metrics (e.g. number of code changes) derived from the source code repository
yield to be best predictor. 
On the other hand Wolf et al.~\cite{wolf:icse:2009} showed that communication structure has an influence on the build outcome.
Kwan et al~\cite{kwan:tse:2011} investigated socio-technical networks with an emphasis on technical dependencies among developers that is not accompanied with any coordination activity on build outcome (socio-technical gap).
We complement the work of Kwan et al's ~\cite{kwan:tse:2011} that showed a relationship between the number of gaps and build failure, by leveraging information about these gaps to create actionable knowledge.

\paragraph{Coordination in software development}
In order to manage changes and maintain quality, developers must coordinate. In
software development, coordination is largely achieved by communication with those sharing work dependencies~\cite{kraut1995:coordination}. 
The software engineering literature is recognizing the role of communication as essential~\cite{nakakoji2010:rdc}.
Several researchers in the software engineering community investigated the effects of communication on several topics such as knowledge distribution~\cite{ehrlich:icgse:2006}, coordination~\cite{hinds:cscw:2006}, and Conway's Law~\cite{cataldo:cscw:2006}.
We build onto these findings in devising an approach that improves team communication.


\paragraph{Coordination and failure in software development}
Many studies bring evidence of the relationship between coordination and software integrations. Besides qualitative studies (e.g~\cite{herbsleb:icse:1999}), quantitative approaches often represent coordination among developers in the form of social networks. These social networks, when related to actual code artifacts, can be used to predict the artifacts' failure-likelihood.
Several studies showed that metrics derived from social networks form good failure predictors (e.g.~\cite{meneely:fse:2008}).
Furthermore, when combining these social networks with information of organizational hierarchies~\cite{nagappan:icse:2008}, or geographical distance~\cite{bird:acm:2009}, they yield not only better predictors but also shine a light on other factors that influence developers' coordination.
% herbsleb grinter
%Herbselb and Grinter~\cite{herbsleb:icse:1999} found that coordination impacts integration efforts.
%We take the findings that show the influence of coordination on software quality and combine them with technical dependencies among developers to generate actionable knowledge.

Recent work has studied the effects of socio-technical alignment on coordination and project outcomes. The mismatch between coordination needs and actual coordination -- implied by work dependencies and developer social interactions respectively -- is referred to as a socio-technical gap. Empirical studies related socio-technical gaps to low developer productivity~\cite{valetto:msr:2007} and low software quality~\cite{kwan:tse:2011}.
With respect to software builds, Kwan et al~\cite{kwan:tse:2011} investigated the occurrence of socio-technical gaps throughout builds in the Rational Team Concert project highlighting that builds that contain more gaps are more likely to fail.
%Valetto et al~\cite{valetto:msr:2007} uncovered a similar insight into productivity related to implementing work items showing that the existence of gaps is a major cause for low productivity.
These findings suggest that the mismatch between implied coordination needs and actual coordination should be avoided to both guarantee high productivity and software quality.



%\subsection{Socio-Technical Congruence}
%Social-technical congruence as originally observed by Conway~\cite{conway:datamination:1968} states that any product developed by an organization will inevitably mirror the organization's communication structure.
%From this starting point Cataldo et al~\cite{cataldo:cscw:2006} as well as other researchers~\cite{valetto:msr:2007,ducheneaut:cscw:2005,ehrlich:stc:2008} investigated whether the lack of this reflection relates to changes in productivity by investigating the overlap of communication among developers and their technical dependencies.
%The communication among developers represents the organizational communication structure whereas the technical dependencies between the work each developer represents the products organization.
%If the communication structure completely contains the work dependencies among developers, then developers accomplish their tasks faster for reasons that are mainly due to knowledge seeking and sharing~\cite{desouza2006:knowledge}.
%For example, a developer can better accomplish her task if she is talking directly to co-workers that need to modify related code to avoid failures or because someone can help her understand the impact the code she is about to modify better.





%\subsection{Recommendations in Software Engineering}
%In the software engineering community knowledge extracted from software repositories is usually brought to developers in the form or recommender systems.
%Several recommender systems derived from the implication of socio-technical congruence described by Conway's Law~\cite{conway:datamination:1968} provide additional awareness to improve coordination among software development especially in a distributed setting where coordination is most difficult~\cite{olson:hci:2000}.
%In the following we describe three exemplary awareness systems.
%
%% tesseract
%\emph{Tesseract}~\cite{sarma:icse:2009} leverages code dependencies among developers to foster awareness through visualizing task and developer centric socio-technical networks.
%A task centric socio-technical network is build from all developers and source code changes that are related through code dependencies or task discussions.
%These task centric socio-technical networks are complemented by developer centric networks, that show for a specific developer what  socio-technical relationships she has with her colleagues.
%
%% proxi scentia
%Systems like Tesseract suffer from the issue that they cannot provide real time feedback on changes in  technical networks, as they solely rely changes committed to the source code repository. 
%\emph{Proxiscentia}~\cite{borici:chase:2012} address this issue by implementing an approach proposed by Blincoe et al~\cite{blincoe:cscw:2012} to instrument IDE's used by software developers and gather code edit events as recorded by tools such as Mylyn~\cite{kersten:aosd:2005}.
%
%% Ensemble
%\emph{Ensemble}~\cite{xiang:rsse:2008} provides a constant stream of events consisting of modifications to artifacts that are related to the stream owner.
%For example, if developer Adam posts a comment on a task owned by developer Eve, then Eve's stream would contain an event showing that Adam commented on her task.
%
%%remarks
%Overall these recommender systems provide awareness of who might be worth to interact with.
%None of those systems are aiming at a concrete goal to accomplish other than achieving awareness.
%We think that a focus is needed, such as on awareness with respect to dependencies that are relevant for build success.
%Without such a focus the information that a developer needs to survey can quickly take up to much precious development time and may lead a developer to abandon those systems as they are taking up more time than they save.
%









