% !TEX root = thesis-journal.tex
\section{Background}
\label{chap:bg}
In this chapter we provide an overview of five areas that are relevant to the research conducted with this thesis: (1) the research on software builds, (2) the research on coordination in software development teams (3) the research around the concept of socio-technical congruence, (4) failure prediction using social networks, and (5) recommender systems in software engineering.

\subsection{Coordination in Software Engineering Teams}
Software is extremely complex because of the sheer number of dependencies~\cite{sawyer2004:teams}.
Large software projects have a large number of components that interoperate with one another.
The difficulty arises when changes must be made to the software, because a change in one component of the software often requires changes in dependent components~\cite{desouza:2008}. Because a single person's knowledge of a system is specialized as well as limited, that person often is unable to make the appropriate modifications in dependent components when a component is changed.

Coordination is defined as ``integrating or linking together different parts of an organization to accomplish a collective set of tasks''~\cite{vandeven1976}. In order to manage changes and maintain quality, developers must coordinate, and in software development, coordination is largely achieved by communicating with people who depend on the work that you do \cite{kraut:1995coordination}.

A successful software build can be viewed as the outcome of good coordination because the build requires the correct compilation of multiple, dependent files of source code.
A failed build, on the other hand, demotivates software developers \cite{holck2004,damian:icgse:2007} and destabilizes the product \cite{cusumano1997}.
While a failed build is not necessarily a disaster, it slows down work significantly while developers scramble to repair the issues.
A build result thus serves as an indicator of the health of the software project up until that point in time.

Thus, a developer should coordinate closely with individuals whose technical dependencies affect his work in order to effectively build software. This brings forth the idea of aligning the technical structure and the social interactions \cite{herbsleb2007:fose}, leading us to the foundation of socio-technical congruence.

Research in software-engineering coordination has examined interactions among
software developers \cite{carter2004,marczak:re:2008}, how they acquire
knowledge \cite{ehrlich:icgse:2006,nakakoji2010:rdc}, and
how they cope with issues including geographical
separation~\cite{espinosa2007:team_knowledge,herbsleb2003:speed}.
The ability to coordinate has
been shown as an influential factor in customer satisfaction \cite{kraut:1995coordination} and  improves the capability to produce quality work~\cite{faraj2000}.


Software developers spend much of their time
communicating~\cite{perry94}. Because developers face
problems when integrating different components from heterogeneous environments~\cite{redmiles2007:continuous},
developers engage in direct or indirect
communication, either to coordinate their activities, or to acquire knowledge of
a particular aspect of the software ~\cite{nakakoji2010:rdc}.
Herbsleb, et al examined the influence of coordination on integrating software
modules through interviews~\cite{herbsleb1999:architectures}, and found that
processes, as well as the willingness to communicate directly, helped teams
integrate software. De Souza et al~\cite{desouza2007:awarenessnetwork} found that implicit
communication is important to avoid collaboration breakdowns and delays. Ko et al~\cite{ko:icse:2007} found that developers were identified as the main source of knowledge about code issues.
Wolf et al~~\cite{wolf:icse:2009} used properties of social networks to predict the outcome of integrating the software parts within teams.
This prior work establishes the fact that developers communicate heavily about technical matters.

Coordinating software teams becomes more difficult as the distance between people increases \cite{herbsleb:icse:2001}.
Studies of Microsoft~\cite{bird2009:dds_quality,nagappan:icse:2008}
show that distance between people that work together on a
program determine the program's failure proneness.
Differences in time zones can affect the number of defects in software projects \cite{cataldo2009:quality}.

Although distance has been identified as a challenge, advances in collaborative
development environments are enabling people to overcome challenges of distance.
One study of early RTC development
shows that the task completion time is not as strongly affected by distance as in previous studies~\cite{Nguyen:2008Distance}. Technology that empowers distributed collaboration include topic recommendations~\cite{carter2004} and instant messaging~\cite{niinimaki2008}. Processes are adapting to the fast pace of software development: the Eclipse way~\cite{frost:ieeesoftware:2007} emphasizes placing milestones at fixed intervals and community involvement.
These new processes lie the Eclipse way that focus on frequent milestones lends more importance to software builds warranting more support by research as we conduct it in this thesis.

\subsection{Can communication predict build failure?}
\label{sec:ResearchQuestions}
Social network analysis has an extensive body of knowledge of analysis and implications with respect to communication and knowledge management
processes~\cite{Burt:1995vo,Freeman:1979rl}. Griffin and
Hauser~\cite{Griffin:1992ms} investigated social networks in manufacturing teams.
They found that a higher connectivity between engineering and marketing increases
the likelihood of a successful product. Similarly, Reagans and
Zuckerman~\cite{RayReagans:2001os} related higher perceived outcomes to denser
communication networks in a study of research and development teams.

Communication structure in particular -- the topology of a communication network
-- has been studied in relation to coordination
(e.g.~\cite{hossain:cscw:2006,hinds:cscw:2006}) and a number of common measures of
communication structure include network density, centrality and structural
holes~\cite{Wasserman:1994sq,Freeman:1979rl}.

Density, as a measure of the extent to which all members in a team are
connected to one another, reflects the ability to distribute
knowledge~\cite{Rulke:2000ys}. Density has been studied, for example, in relation
to coordination ease~\cite{hinds:cscw:2006}, coordination
capability~\cite{hossain:cscw:2006} and enhanced group
identification~\cite{RayReagans:2001os}.

Centrality measures indicate importance or prominence of actors in a
social network. The most commonly used centrality measures include degree and
betweenness centrality having different social implication. Centrality measures
have been used to characterize and compare different communication networks
constructed from email correspondence of W3C (WWW consortium) collaborating
working groups developing new technical standards and architectures for the
web~\cite{Gloor:2003cikm}. Similarly, Hossain et al~\cite{hossain:cscw:2006}
explored the correlation between centrality in email-based communication networks
and coordination, and found betweenness to be the best measure for coordination.
Betweenness is a measure of the extent to which a team member is
positioned on the shortest path in between other two members. People in between
are considered to be ``actors in the middle'' and to have more ``interpersonal
influence'' in the
network(e.g.~\cite{Gloor:2003cikm,zimmermann:icse:2008,hossain:cscw:2006}).

The structural holes measures are concerned with the degree to which there
are missing links in between nodes and with the notion of redundancy in
networks~\cite{Burt:1995vo}. At the node level, structural holes are gaps between
nodes in a social network. At the network level, people on either side of the
hole have access to different flows of information~\cite{Hargadon:1997asq},
indicating that there is a diversity of information flow in the network.
Structural holes have been used to measure social capital in relation to the
performance of academic collaborators (e.g.~\cite{Brambila:PICMET2007}).

Most prediction models in software engineering to date mainly leverage source
code related data and focus on predicting failing software components or failure
inducing changes
(e.g.~\cite{bell:2005tse,schroeter:isese:2006,zimmermann:icse:2008,kim:2008tse}).
And only few studies, such as Hassan and Zhang~\cite{hassan:ase:2006}, stepped away
from predicting component failures and used statistical classifiers to predict
integration outcome.
In this thesis we want to extend the body of knowledge surrounding prediction models using communication data or focusing on build outcome by investigating how to improve communication among software developers to prevent build failures.







\subsection{Socio-Technical Congruence}
Social-technical congruence as originally observed by Conway~\cite{conway:datamination:1968} states that any product developed by an organization will inevitably mirror the organization's communication structure.
From this starting point Cataldo et al~\cite{cataldo:cscw:2006} as well as other researchers~\cite{valetto:msr:2007,ducheneaut:cscw:2005,ehrlich:stc:2008} investigated whether the lack of this reflection relates to changes in productivity by investigating the overlap of communication among developers and their technical dependencies.
The communication among developers represents the organizational communication structure whereas the technical dependencies between the work each developer represents the products organization.
If the communication structure completely contains the work dependencies among developers, then developers accomplish their tasks faster for reasons that are mainly due to knowledge seeking and sharing~\cite{desouza2006:knowledge}.
For example, a developer can better accomplish her task if she is talking directly to co-workers that need to modify related code to avoid failures or because someone can help her understand the impact the code she is about to modify better.

%The main performance criteria research investigated to measure the effect of socio-technical congruence is task completion time.
%For this purpose Cataldo et al~\cite{cataldo:cscw:2006} measures the congruence on a task basis and test for the correlation between congruence the metric with the time it took to resolve the task.
%Overall Cataldo et al~\cite{cataldo:cscw:2006} found that there is a statistically significant relation between the amount of congruence and a tasks resolution time, which was confirmed by other studies~\cite{valetto:msr:2007,ehrlich:stc:2008}.




\subsection{Recommendations in Software Engineering}
In the software engineering community knowledge extracted from software repositories is usually brought to developers in the form or recommender systems.
Several recommender systems derived from the implication of socio-technical congruence described by Conway's Law~\cite{conway:datamination:1968} provide additional awareness to improve coordination among software development especially in a distributed setting where coordination is most difficult~\cite{olson:hci:2000}.
In the following we describe three exemplary awareness systems.

% ariadne
%\emph{Ariadne}~\cite{trainer2005:ariadne} provides awareness to developers by showing call dependencies between code a developer is working on and the code that she is potentially affecting.
%This allows a developer to see which other developer she might need to coordinate her work with to not negatively impact that developer's code.

% palantir
%\emph{Palantir}~\cite{sarma:cscw:2002} complements the dependencies among developers by providing the reverse awareness  showing a developer what source code she is currently accessing in their workspace is affected by code changes submitted by co-workers.
%For example, Palantir indicates which source code files have been changed in the mean time by other developers that are present in the developer's current work space and thus might hint at possible merge conflicts.

% tesseract
\emph{Tesseract}~\cite{sarma:icse:2009} leverages code dependencies among developers to foster awareness through visualizing task and developer centric socio-technical networks.
A task centric socio-technical network is build from all developers and source code changes that are related through code dependencies or task discussions.
These task centric socio-technical networks are complemented by developer centric networks, that show for a specific developer what  socio-technical relationships she has with her colleagues.

% proxi scentia
Systems like Tesseract suffer from the issue that they cannot provide real time feedback on changes in  technical networks, as they solely rely changes committed to the source code repository. 
\emph{Proxiscentia}~\cite{borici:chase:2012} address this issue by implementing an approach proposed by Blincoe et al~\cite{blincoe:cscw:2012} to instrument IDE's used by software developers and gather code edit events as recorded by tools such as Mylyn~\cite{kersten:aosd:2005}.

% Ensemble
\emph{Ensemble}~\cite{xiang:rsse:2008} provides a constant stream of events consisting of modifications to artifacts that are related to the stream owner.
For example, if developer Adam posts a comment on a task owned by developer Eve, then Eve's stream would contain an event showing that Adam commented on her task.

%remarks
Overall these recommender systems provide awareness of who might be worth to interact with.
None of those systems are aiming at a concrete goal to accomplish other than achieving awareness.
We think that a focus is needed, such as on awareness with respect to dependencies that are relevant for build success.
Without such a focus the information that a developer needs to survey can quickly take up to much precious development time and may lead a developer to abandon those systems as they are taking up more time than they save.





\section{Research Questions}
The concept of socio-technical congruence shows potential to help make software development more efficient.
Cataldo et al~\cite{cataldo:cscw:2006} demonstrated its relation to productivity, and we show among other things in this thesis the ability to use socio-technical congruence to predict build outcome.
The concept of socio-technical congruence lends itself to improve software development as it is based on social networks connecting developer on a coordination and technical level.
Because of the concept being based on networks it is possible to manipulate the networks.

Any socio-technical network can be manipulated in two ways: (1) changing the technical dependencies among developer by refactoring or architectural changes to make them unnecessary and (2) by engaging developer in discussions about their recent work and therefore creating a coordination edge in the socio-technical network.
Since many products are not developed from scratch and because architectural changes once development has been going on for a number of months are costly and time consuming~\cite{vangurp:jss:2002}, we aim at generating recommendations to change the actual coordination to improve the socio-technical network where it matters.
Therefore, as a first step we need to assess if the actual communication structure among software developers has an influence on build success to lay the basis for manipulating the actual coordination to increase build success.
As a follow up step, we need to explore the relationship between socio-technical networks and build success.
Especially we are interested in whether missing actual coordination in the face of a coordination needs is related to build failure.

We start in the second part of this thesis with investigating the influence of communication among team members in the form of social networks on build success.
Next, we investigate if gaps (unfilled coordination needs) between developers as highlighted by socio-technical networks and the socio-technical networks themselves can be brought into relation with build success.
Therefore Chapter~\ref{chap:soc-net} and~\ref{chap:stc-net2} investigate the following two research questions respectively:

\begin{itemize}
  \item\textbf{RQ 1.1:} Do Social Networks influence build success? (Chapter~\ref{chap:soc-net})
  \item\textbf{RQ 1.2:} Does Socio-Technical Networks influence build success? (Chapter~\ref{chap:stc-net2})
\end{itemize}

Having found a relationship between socio-technical networks, especially gaps between coordination and coordination needs with build success, while knowing that communication alone has an effect on build success, we formulate an approach to leverage socio-technical networks (Chapter~\ref{chap:approach}).
The third and final part of this thesis focuses on evaluating this approach in three ways:
(1) gathering general statistical evidence that parts of the network can be manipulated to increase build success,
(2) exploring the acceptance of such recommendation based on those manipulations by developers,
and (3) a proof of concept that the recommendation could prevent failures.
Hence, the first three chapters of the third part of this thesis are guided by the following three research questions:

\begin{itemize}
  \item\textbf{RQ 2.1:} Can Socio-Technical Networks be manipulated to increase build success? (Chapter~\ref{chap:stc-net})
  \item\textbf{RQ 2.2:} Do developers accept recommendations based on software changes to increase build success? (Chapter~\ref{chap:talk})
\end{itemize}

In the following discussion (Chapter~\ref{chap:disc}) we will highlight how our findings from these three research questions support the approach we detailed in Chapter~\ref{chap:approach}.









